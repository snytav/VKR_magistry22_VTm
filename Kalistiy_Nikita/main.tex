\documentclass{article}

% Language setting
% Replace `english' with e.g. `spanish' to change the document language
\usepackage[english,russian]{babel}

% Set page size and margins
% Replace `letterpaper' with `a4paper' for UK/EU standard size
\usepackage[letterpaper,top=2cm,bottom=2cm,left=3cm,right=3cm,marginparwidth=1.75cm]{geometry}

% Useful packages
\usepackage{amsmath}
\usepackage{graphicx}
\usepackage[colorlinks=true, allcolors=blue]{hyperref}

\title{Анализ поведения пользователей в текстовых ассистентах и выработка методологий оптимального проектирования пользовательского интерфейса и сценариев}
\author{Калистый Н.С.}

\begin{document}
\maketitle

\begin{abstract}
Данная исследовательская работа представляет собой важный вклад в активно развивающееся направление программирования и разработки решений для автоматизации бизнес-процессов с использованием текстовых чат-ботов. В современном мире чат-боты становятся все более популярными благодаря своей универсальности и доступности, не требуя от пользователей установки дополнительных приложений.
Однако, несмотря на актуальность данной темы, практические исследования в этой области остаются недостаточно распространенными. Исследование предлагает оценить эффективность текстовых чат-ботов для автоматизации бизнес-процессов на основе анализа фактических данных, полученных в результате работы трех выпущенных в релиз продуктов. Эмпирический метод исследования, основанный на анализе логов приложения, позволяет оценить реальное воздействие чат-ботов на пользователей.
\end{abstract}

\section{Введение}
Современные технологии информационной обработки и обмена данными значительно изменили повседневную жизнь и бизнес-процессы. В наше время цифровые помощники, такие как текстовые чат-боты, стали неотъемлемой частью взаимодействия между человеком и информацией. Эти боты стали мощным средством автоматизации бизнес-процессов и обеспечения качественного обслуживания клиентов.

В данной работе рассматривается актуальная тема анализа поведения пользователей в текстовых ассистентах и разработки методологий оптимального проектирования интерфейсов и сценариев. Анализ поведения пользователей означает более глубокое понимание того, как пользователи взаимодействуют с ботами, какие запросы они делают, какие ошибки допускают, и в каких местах интерфейса они испытывают затруднения.

Однако, разработка эффективных и удовлетворяющих потребности пользователей чат-ботов включает в себя не только анализ, но и активное участие в процессе проектирования интерфейсов и сценариев. Это означает формирование набора правил и рекомендаций, которые определяют, как бот будет взаимодействовать с пользователями, какие сообщения он будет отправлять, какие кнопки и элементы управления будет использовать, и многое другое.

Данная работа нацелена на изучение этой сложной и многогранной проблематики на примере разработки приложений для автоматизации бизнес-процессов с использованием текстовых ассистентов, доступных через Telegram. В этом контексте, исследование включает в себя как анализ поведения пользователей, так и разработку методологий оптимального проектирования интерфейсов и сценариев. Кроме того, данная работа также обращает внимание на использование нейронных сетей для интеллектуального анализа пользовательского ввода и улучшения качества обслуживания.
\section{Актуальность}
Интернет-сервисы и бизнес-приложения с текстовыми ассистентами стали важной составной частью современного информационного ландшафта. Они используются в таких областях, как клиентский сервис, маркетинг, финансы, здравоохранение и многие другие. Но для того чтобы сделать их более эффективными и удовлетворяющими потребности пользователей, необходимо более глубокое понимание того, как пользователи с ними взаимодействуют и какие проблемы они могут столкнуться.

В настоящее время использование чат-ботов становится особенно актуальным как для обычных пользователей, так и для предприятий и бизнеса. Это обусловлено несколькими факторами, которые делают чат-боты важными и выгодными.

Во-первых, для обычных пользователей использование чат-ботов представляет собой удобное и легкодоступное средство общения с различными сервисами и организациями. Одним из ключевых преимуществ чат-ботов является то, что они не требуют установки дополнительных приложений на устройства пользователя. Это делает процесс взаимодействия максимально удобным и доступным.

Во-вторых, чат-боты обладают широким спектром функциональных возможностей. Они способны автоматизировать разнообразные задачи, начиная от оформления заказов и предоставления консультаций, и заканчивая решением сложных бизнес-процессов. Это позволяет как обычным пользователям, так и бизнес-организациям значительно сэкономить время и ресурсы.

Следует отметить, что для бизнеса, особенно для малых и средних предприятий, чат-боты становятся настоящими инструментами экономии. Разработка и сопровождение текстовых ассистентов оказывается гораздо более дешевой альтернативой по сравнению с созданием и постоянным обновлением мобильных приложений под разные операционные системы, такие как Android и iOS.
Таким образом, чат-боты сегодня являются неотъемлемой частью современной информационной среды, обеспечивая удобство, функциональность и экономичность как для обычных пользователей, так и для предприятий и бизнеса.
\section{Новизна}
Данное исследование отличается от большинства исследований в области разработки чат-ботов тем, что оно не ограничивается конкретной платформой для создания ботов. Вместо этого, оно предоставляет общие рекомендации, которые будут полезны всем разработчикам чат-ботов, независимо от используемых инструментов. Это делает работу более универсальной и применимой для широкого круга специалистов в области разработки.

Важно подчеркнуть, что в данной работе понятие "чат-бот" рассматривается в широком смысле. Чат-боты рассматриваются как полноценные бэкенд-приложения, способные автоматизировать сложные бизнес-процессы. Они служат средством снижения затрат бизнеса на взаимодействие с клиентами, улучшения качества обслуживания и оптимизации рабочих процессов.

В данной работе рассматривается достаточно сложный функционал, который типичные чат-боты обычно не включают в свой набор возможностей. В работе рассматривается интеллектуальный анализ пользовательского ввода с использованием нейросетей, анализ поведения пользователей в приложении и оптимизацию сценариев взаимодействия. Такой подход делает исследование особенно ценным для разработчиков, стремящихся создать более сложные и функциональные чат-боты, способные решать более широкий спектр задач.

Важным аспектом новизны данной работы является акцент на практических результатах. Анализируемые в работе данные представляют собой информацию о реальных пользователях и клиентах бизнеса. Полученные результаты и рекомендации основаны на фактических данных о поведении пользователей, что делает исследование не только актуальным, но и практически применимым для компаний, занимающихся разработкой и внедрением чат-ботов в бизнес-процессы.

В целом, данное исследование представляет собой уникальный вклад в область разработки чат-ботов, который охватывает широкий спектр тем и предоставляет практические рекомендации для улучшения качества и эффективности чат-ботов как инструмента для автоматизации бизнес-процессов.
\section{Обзор литературы}
В работе \cite{1} сделано все плохо.



\section{Описание работы}
1. Сбор и обработка данных:

Для начала проекта необходимо обзавестись исходными данными - логами приложений. Мы используем программу аналитики работы приложений под названием TockBot для сбора данных. Логи приложения представляют собой массив JSON-файлов, каждая строка которых содержит ценную информацию. В каждой строке фиксируется идентификатор бота, в котором произошло событие, идентификатор пользователя, выполнившего действие, ввод, совершенный им, а также информация о состояниях (вершинах графа), между которыми произошел переход.

Лог также включает ответ чат-бота, который может содержать текстовые сообщения, кнопки, причем в анализируемом примере чат ботов в Telegram чат-боты поддерживают два вида кнопок: Inline, которые прикрепляются к сообщению, и Reply, которые отображаются внизу графического интерфейса, под полем текстового ввода и служат пользователю для быстрого ответа боту без необходимости писать ввод вручную.

Лог также включает отправленные ботом изображения, документы и другие реакции. Важным атрибутом каждого лога является timestamp, который фиксирует дату и время события. Для анализа планируется использовать несколько тысяч записей из трех различных проектов, уже запущенных в работу и имеющих данные от реальных пользователей.

Логи серверной части приложения во время его работы могут выглядеть таким образом.
Логи серверной части приложения во время его работы могут выглядеть таким образом.

\item{
{"timestamp":"2023-10-18 15:02:17,560","level":" INFO","from":"com.justai.jaicf.logging.Slf4jConversationLogger","message":{"userId":"587692838","processing":"TelegramQueryRequest","input":"↪️ На шаг назад","reactions":[{"say":{"text":"↪️ Переход назад"}},{"edit_message_reply_markup":{"displayed_text":"↪️ Переход назад"}},{"go":{"from_state":"/admin/admin_create_application_for_user/select_product/go_back","to_state":"/admin/admin_create_application_for_user"}},{"say":{"text":"Для поиска контрагентов вы можете:\n\n— <i>написать номер телефона, в формате 11 цифр, без плюса, скобок, дефисов и прочих символов</i>\n— <i>отправить контакт пользователя чат-бота, которого хотите записать</i>\n\n<a href=\"https://wiki.conversationlab.ru/ru/projects/VelodyBot\">Инструкция</a>"}}],"fromState":"/admin/admin_create_application_for_user"}}
}
\item{
{"timestamp":"2023-10-18 15:02:24,855","level":" INFO","from":"com.justai.jaicf.logging.Slf4jConversationLogger","message":{"userId":"587692838","processing":"TelegramTextRequest","input":"79854123541","reactions":[{"say":{"text":"Произвожу поиск, пожалуйста, подождите..."}},{"edit_message_text":{"new_text":"Контрагент с таким номером телефона не найден\n\nВы можете:\n<i>— продолжить оформление услуги\n— написать другой номер телефона\n— отправить контакт пользователя чат-бота, которого хотите записать</i>\n\n <a href=\"https://wiki.conversationlab.ru/ru/projects/VelodyBot\">Инструкция</a>","message_id":7167}},{"edit_message_text":{"new_text":"Отредактировали сообщение: userId = 587692838, messageId = 7167, inlineMessageId = null, newText =  Контрагент с таким номером телефона не найден\n\nВы можете:\n<i>— продолжить оформление услуги\n— написать другой номер телефона\n— отправить контакт пользователя чат-бота, которого хотите записать</i>\n\n <a href=\"https://wiki.conversationlab.ru/ru/projects/VelodyBot\">Инструкция</a>"}}],"fromState":"/admin/admin_create_application_for_user/check_user_phone_number"}}
}

\item{
{"timestamp":"2023-10-18 15:02:27,505","level":" INFO","from":"com.justai.jaicf.logging.Slf4jConversationLogger","message":{"userId":"587692838","processing":"TelegramQueryRequest","input":"➕ Создать контрагента без ПД","reactions":[{"delete_message":{"user_id":587692838,"message_id":7167}},{"go":{"from_state":"/admin/admin_create_application_for_user/check_user_phone_number/correct_register_user_without_passport","to_state":"/admin/admin_create_application_for_user/add_contractor_without_passport"}},{"go":{"from_state":"/admin/admin_create_application_for_user/add_contractor_without_passport","to_state":"/fillingForm/Filling_form"}},{"say":{"text":"✏️ Внесите фамилию и имя человека с номером телефона <code>79854123541</code>, чтобы добавить контрагента без ПД"}}],"fromState":"/fillingForm/Filling_form"}}
}

\item{
{"timestamp":"2023-10-18 15:02:29,127","level":" INFO","from":"com.justai.jaicf.logging.Slf4jConversationLogger","message":{"userId":"587692838","processing":"TelegramQueryRequest","input":"✏️ Заполнить форму","reactions":[{"say":{"text":" Заполняйте данные\n\nНажимайте:\n\n«◀️» и «▶️» для перехода по вопросам анкеты\n<b>«✅ Отправить»</b> для отправки анкеты\n<b>« Заполнить заново»</b> для заполнения анкеты заново\n<b>«❌ Отменить»</b> для отмены регистрации"}},{"go":{"from_state":"/fillingForm/Filling_form/start_filling","to_state":"/fillingForm/Filling_form/continue"}},{"say":{"text":" <b>1/2</b>\n\n⚠️ Клиент должен предоставить имя и фамилию, чтобы зарегистрироваться без ПД и начать оформление услуги\n\n✍️ Напишите <i>фамилию клиента</i>\n\nНапример:\n\n<i>Иванов</i>\n<i>Сергеев</i>"}},{"edit_message_text":{"new_text":"вместо редактирования сообщения отправили новое сообщение с текстом ->  <b>1/2</b>\n\n⚠️ Клиент должен предоставить имя и фамилию, чтобы зарегистрироваться без ПД и начать оформление услуги\n\n✍️ Напишите <i>фамилию клиента</i>\n\nНапример:\n\n<i>Иванов</i>\n<i>Сергеев</i>"}}],"fromState":"/fillingForm/Filling_form/continue"}}
}

\item{
{"timestamp":"2023-10-18 15:02:31,227","level":" INFO","from":"com.justai.jaicf.logging.Slf4jConversationLogger","message":{"userId":"587692838","processing":"TelegramTextRequest","input":"ertujdtr","reactions":[{"say":{"text":"✏️ Записал ваш ответ: <b>ertujdtr</b>"}},{"go":{"from_state":"/fillingForm/Filling_form/continue/check_text_input","to_state":"/fillingForm/Filling_form/continue/next"}},{"go":{"from_state":"/fillingForm/Filling_form/continue/next","to_state":"/fillingForm/Filling_form/continue"}},{"say":{"text":" <b>2/2</b>\n\n✍️ Напишите <i>имя клиента</i>\n\nНапример:\n\n<i>Иван</i>\n<i>Сергей</i>"}},{"edit_message_text":{"new_text":"вместо редактирования сообщения отправили новое сообщение с текстом ->  <b>2/2</b>\n\n✍️ Напишите <i>имя клиента</i>\n\nНапример:\n\n<i>Иван</i>\n<i>Сергей</i>"}}],"fromState":"/fillingForm/Filling_form/continue"}}
}

\item{
{"timestamp":"2023-10-18 15:02:36,667","level":" INFO","from":"com.justai.jaicf.logging.Slf4jConversationLogger","message":{"userId":"587692838","processing":"TelegramTextRequest","input":"edjedtj","reactions":[{"say":{"text":"✏️ Записал ваш ответ: <b>edjedtj</b>"}},{"go":{"from_state":"/fillingForm/Filling_form/continue/check_text_input","to_state":"/fillingForm/Filling_form/continue/offer_finish_filling"}},{"go":{"from_state":"/fillingForm/Filling_form/continue/offer_finish_filling","to_state":"/fillingForm/Filling_form/continue/finish"}},{"say":{"text":"✅ Форма успешно заполнена!"}},{"edit_message_reply_markup":{"displayed_text":"✅ Форма успешно заполнена!"}},{"go":{"from_state":"/fillingForm/Filling_form/continue/finish","to_state":"/admin/admin_create_application_for_user/add_contractor_without_passport/correct"}},{"say":{"text":"⏳ Создаю контрагента, ожидайте..."}},{"edit_message_text":{"new_text":"✅ Успешно создан контрагент <i><i>ertujdtr edjedtj</i></i> с номером телефона <code>79854123541</code>\n\nПереходим к выбору услуги","message_id":7177}},{"edit_message_text":{"new_text":"Отредактировали сообщение: userId = 587692838, messageId = 7177, inlineMessageId = null, newText = ✅ Успешно создан контрагент <i><i>ertujdtr edjedtj</i></i> с номером телефона <code>79854123541</code>\n\nПереходим к выбору услуги"}},{"go":{"from_state":"/admin/admin_create_application_for_user/add_contractor_without_passport/correct","to_state":"/admin/admin_create_application_for_user/select_product"}},{"say":{"text":" Сейчас я проверю данные контрагента и мы перейдем к выбору услуг, пожалуйста, подождите..."}},{"edit_message_text":{"new_text":" Выберите услугу, которую хотите оформить для клиента <i>ertujdtr edjedtj </i>\n\nСтатус контрагента: <i>зарегистрирован с ПД</i>","message_id":7178}},{"edit_message_text":{"new_text":"Отредактировали сообщение: userId = 587692838, messageId = 7178, inlineMessageId = null, newText =  Выберите услугу, которую хотите оформить для клиента <i>ertujdtr edjedtj </i>\n\nСтатус контрагента: <i>зарегистрирован с ПД</i>"}}],"fromState":"/admin/admin_create_application_for_user/select_product"}}
}

\item{
{"timestamp":"2023-10-18 15:06:56,097","level":" INFO","from":"com.justai.jaicf.logging.Slf4jConversationLogger","message":{"userId":"587692838","processing":"TelegramQueryRequest","input":"↪️ На шаг назад","reactions":[{"say":{"text":"↪️ Переход назад"}},{"edit_message_reply_markup":{"displayed_text":"↪️ Переход назад"}},{"go":{"from_state":"/admin/admin_create_application_for_user/select_product/go_back","to_state":"/admin/admin_create_application_for_user"}},{"say":{"text":" Для поиска контрагентов вы можете:\n\n— <i>написать номер телефона, в формате 11 цифр, без плюса, скобок, дефисов и прочих символов</i>\n— <i>отправить контакт пользователя чат-бота, которого хотите записать</i>\n\n <a href=\"https://wiki.conversationlab.ru/ru/projects/VelodyBot\">Инструкция</a>"}}],"fromState":"/admin/admin_create_application_for_user"}}
}

\item{
{"timestamp":"2023-10-18 15:07:02,991","level":" INFO","from":"com.justai.jaicf.logging.Slf4jConversationLogger","message":{"userId":"587692838","processing":"TelegramTextRequest","input":"78452651843","reactions":[{"say":{"text":" Произвожу поиск, пожалуйста, подождите..."}},{"edit_message_text":{"new_text":" Контрагент с таким номером телефона не найден\n\nВы можете:\n<i>— продолжить оформление услуги\n— написать другой номер телефона\n— отправить контакт пользователя чат-бота, которого хотите записать</i>\n\n <a href=\"https://wiki.conversationlab.ru/ru/projects/VelodyBot\">Инструкция</a>","message_id":7182}},{"edit_message_text":{"new_text":"Отредактировали сообщение: userId = 587692838, messageId = 7182, inlineMessageId = null, newText =  Контрагент с таким номером телефона не найден\n\nВы можете:\n<i>— продолжить оформление услуги\n— написать другой номер телефона\n— отправить контакт пользователя чат-бота, которого хотите записать</i>\n\n <a href=\"https://wiki.conversationlab.ru/ru/projects/VelodyBot\">Инструкция</a>"}}],"fromState":"/admin/admin_create_application_for_user/check_user_phone_number"}}
}

\item{
{"timestamp":"2023-10-18 15:07:04,625","level":" INFO","from":"com.justai.jaicf.logging.Slf4jConversationLogger","message":{"userId":"587692838","processing":"TelegramQueryRequest","input":"➕ Создать контрагента без ПД","reactions":[{"delete_message":{"user_id":587692838,"message_id":7182}},{"go":{"from_state":"/admin/admin_create_application_for_user/check_user_phone_number/correct_register_user_without_passport","to_state":"/admin/admin_create_application_for_user/add_contractor_without_passport"}},{"go":{"from_state":"/admin/admin_create_application_for_user/add_contractor_without_passport","to_state":"/fillingForm/Filling_form"}},{"say":{"text":"✏️ Внесите фамилию и имя человека с номером телефона <code>78452651843</code>, чтобы добавить контрагента без ПД"}}],"fromState":"/fillingForm/Filling_form"}}
}

\item{
{"timestamp":"2023-10-18 15:07:05,782","level":" INFO","from":"com.justai.jaicf.logging.Slf4jConversationLogger","message":{"userId":"587692838","processing":"TelegramQueryRequest","input":"✏️ Заполнить форму","reactions":[{"say":{"text":" Заполняйте данные\n\nНажимайте:\n\n«◀️» и «▶️» для перехода по вопросам анкеты\n<b>«✅ Отправить»</b> для отправки анкеты\n<b>« Заполнить заново»</b> для заполнения анкеты заново\n<b>«❌ Отменить»</b> для отмены регистрации"}},{"go":{"from_state":"/fillingForm/Filling_form/start_filling","to_state":"/fillingForm/Filling_form/continue"}},{"say":{"text":" <b>1/2</b>\n\n⚠️ Клиент должен предоставить имя и фамилию, чтобы зарегистрироваться без ПД и начать оформление услуги\n\n✍️ Напишите <i>фамилию клиента</i>\n\nНапример:\n\n<i>Иванов</i>\n<i>Сергеев</i>"}},{"edit_message_text":{"new_text":"вместо редактирования сообщения отправили новое сообщение с текстом ->  <b>1/2</b>\n\n⚠️ Клиент должен предоставить имя и фамилию, чтобы зарегистрироваться без ПД и начать оформление услуги\n\n✍️ Напишите <i>фамилию клиента</i>\n\nНапример:\n\n<i>Иванов</i>\n<i>Сергеев</i>"}}],"fromState":"/fillingForm/Filling_form/continue"}}
}

\item{
{"timestamp":"2023-10-18 15:07:07,185","level":" INFO","from":"com.justai.jaicf.logging.Slf4jConversationLogger","message":{"userId":"587692838","processing":"TelegramTextRequest","input":"вары","reactions":[{"say":{"text":"✏️ Записал ваш ответ: <b>вары</b>"}},{"go":{"from_state":"/fillingForm/Filling_form/continue/check_text_input","to_state":"/fillingForm/Filling_form/continue/next"}},{"go":{"from_state":"/fillingForm/Filling_form/continue/next","to_state":"/fillingForm/Filling_form/continue"}},{"say":{"text":" <b>2/2</b>\n\n✍️ Напишите <i>имя клиента</i>\n\nНапример:\n\n<i>Иван</i>\n<i>Сергей</i>"}},{"edit_message_text":{"new_text":"вместо редактирования сообщения отправили новое сообщение с текстом ->  <b>2/2</b>\n\n✍️ Напишите <i>имя клиента</i>\n\nНапример:\n\n<i>Иван</i>\n<i>Сергей</i>"}}],"fromState":"/fillingForm/Filling_form/continue"}}
}

\item{
{"timestamp":"2023-10-18 15:07:15,279","level":" INFO","from":"com.justai.jaicf.logging.Slf4jConversationLogger","message":{"userId":"587692838","processing":"TelegramTextRequest","input":"ары","reactions":[{"say":{"text":"✏️ Записал ваш ответ: <b>ары</b>"}},{"go":{"from_state":"/fillingForm/Filling_form/continue/check_text_input","to_state":"/fillingForm/Filling_form/continue/offer_finish_filling"}},{"go":{"from_state":"/fillingForm/Filling_form/continue/offer_finish_filling","to_state":"/fillingForm/Filling_form/continue/finish"}},{"say":{"text":"✅ Форма успешно заполнена!"}},{"edit_message_reply_markup":{"displayed_text":"✅ Форма успешно заполнена!"}},{"go":{"from_state":"/fillingForm/Filling_form/continue/finish","to_state":"/admin/admin_create_application_for_user/add_contractor_without_passport/correct"}},{"say":{"text":"⏳ Создаю контрагента, ожидайте..."}},{"edit_message_text":{"new_text":"✅ Успешно создан контрагент <i><i>вары ары</i></i> с номером телефона <code>78452651843</code>\n\nПереходим к выбору услуги","message_id":7192}},{"edit_message_text":{"new_text":"Отредактировали сообщение: userId = 587692838, messageId = 7192, inlineMessageId = null, newText = ✅ Успешно создан контрагент <i><i>вары ары</i></i> с номером телефона <code>78452651843</code>\n\nПереходим к выбору услуги"}},{"go":{"from_state":"/admin/admin_create_application_for_user/add_contractor_without_passport/correct","to_state":"/admin/admin_create_application_for_user/select_product"}},{"say":{"text":" Сейчас я проверю данные контрагента и мы перейдем к выбору услуг, пожалуйста, подождите..."}},{"edit_message_text":{"new_text":" Выберите услугу, которую хотите оформить для клиента <i>вары ары </i>\n\nСтатус контрагента: <i>зарегистрирован без ПД, возможность оформления некоторых услуг ограничена до подачи ПД</i>","message_id":7193}},{"edit_message_text":{"new_text":"Отредактировали сообщение: userId = 587692838, messageId = 7193, inlineMessageId = null, newText =  Выберите услугу, которую хотите оформить для клиента <i>вары ары </i>\n\nСтатус контрагента: <i>зарегистрирован без ПД, возможность оформления некоторых услуг ограничена до подачи ПД</i>"}}],"fromState":"/admin/admin_create_application_for_user/select_product"}}
}

\item{

}

Каждая запись лога представляет собой действие одного пользователя, совершенно в пользовательском интерфейсе, которое приводит к ответу чат-бота в зависимости от логики, прописанной на серверной части приложения. Для удобства анализа данных такого типа, их сводят в тиблицы, например в виде csv формата. 

2. Свод данных в CSV таблицу:

Полученные логи необходимо структурировать и преобразовать в более удобный формат для анализа. Мы сводим все полученные данные в CSV таблицу, что упрощает работу с ними и позволяет проводить более глубокий анализ. Для записи в CSV файл необходимо было реализовать скрипт на языке программирования Python. Текст скрипта.


3. Визуализация графов сценариев:

Для наглядного представления сценариев взаимодействия пользователей с чат-ботами, планируется использовать различные инструменты визуализации данных. Это позволит нам лучше понять структуру чат-бота и выделить основные пути взаимодействия.

4. Анализ данных и их нормализация:

На этом этапе мы проводим анализ данных, включая нормализацию и устранение выбросов. Дополнительно строим карту частот, которая помогает нам выявить популярные пути на графе взаимодействия пользователей. Путем критериальной оценки мы выявляем те пути, которые наиболее востребованы, а также идентифицируем пути, которые можно оптимизировать для улучшения функциональности чат-бота и снижения затрат на его обслуживание.

Корреляционный анализ проводится для определения связей между количеством пользователей, завершивших воронку, и их активностью после этого. Такой анализ может помочь выявить, какие действия пользователей влияют на их долгосрочное использование чат-бота.

5. Исследование с использованием нейронных сетей:

Далее, мы переходим к более глубокому анализу результатов. Используя нейронные сети, мы стремимся выявить общие тенденции в поведении пользователей, их предпочтения и моменты, в которых пользователи отваливаются из воронки. Мы также изучаем сложности, с которыми они сталкиваются, и какие действия могут привести к спаму или нежелательному поведению. Анализируя данные, полученные после анализа нейронными сетями, мы выявляем факторы, которые влияют на успешное завершение воронки.

6. Формулирование гипотез и рекомендаций:

Исходя из результатов исследования, мы формулируем гипотезы и рекомендации. Это включает в себя стратегии по увеличению количества пользователей, завершивших воронку, а также меры для привлечения пользователей, вернувшихся повторно после завершения услуги. Мы также предоставляем практические советы о том, какие методы разработки могут сократить затраты на создание чат-ботов и повысить их понятность для пользователей.

7. Обратная связь и реализация рекомендаций:

Завершая этот проект, мы обмениваемся результатами с заказчиком и получаем обратную связь о результатах, полученных после внедрения предложенных рекомендаций и методологий в исследованные проекты. Это важный этап, который позволяет убедиться в эффективности предложенных мер и внести коррективы в дальнейшей разработке и оптимизации чат-ботов.
\section{Практическая ценность}
Исследование, проведенное в ходе данной работы, имеет высокую практическую ценность как для разработчиков, так и для заказчиков чат-ботов. В процессе исследования были получены ценные инсайты и рекомендации, которые способствуют оптимизации и улучшению сценариев взаимодействия пользователей с чат-ботами.

Одним из важных достижений является установление наиболее востребованных направлений движения пользователей по графу сценариев чат-бота. Это позволило выявить ключевые точки взаимодействия и сделать акцент на них, улучшая пользовательский опыт.

В рамках исследования были разработаны рекомендации по оптимизации сценариев, направленные на сокращение количества вершин графа без потери функциональности приложения чат-бота. Такие оптимизации позволяют сделать интерфейс более интуитивным и удобным для пользователей, снижая вероятность недопонимания и ошибок.

Другим важным результатом исследования стало предложение упрощенных и более ясных формулировок, избегание сложных аббревиатур и сокращений. Эти рекомендации способствуют легкости восприятия информации и обеспечивают четкое взаимодействие пользователей с чат-ботом.

Одним из наиболее заметных достижений данного исследования является разработка методологий, которые позволяют компаниям-заказчикам создавать текстовых ассистентов с учетом всех выявленных рекомендаций. Это позволяет сразу создавать чат-ботов без указанных ошибок, с оптимизированными сценариями и интерфейсом, что сокращает стоимость разработки каждого последующего чат-бота примерно на 15-30%, в зависимости от сложности проекта и числа функциональных возможностей.

Компания-заказчик исследования пришла к положительным выводам от результатов проведенной работы. Полученные рекомендации и методологии уже на практике помогли сократить затраты и сделать взаимодействие с чат-ботами более продуктивным и приятным для пользователей.
\section{Заключение}
Результаты данной работы будут иметь высокую практическую ценность для разработчиков и дизайнеров чат-ботов, а также для бизнес-аналитиков и менеджеров по обслуживанию клиентов. Исследование позволит оптимизировать процессы разработки и существенно улучшить качество обслуживания пользователей через текстовые ассистенты. Кроме того, разработанные методологии могут быть использованы как основа для дальнейших исследований и практических применений в области чат-ботов и искусственного интеллекта.

В ходе проведенного исследования было выявлено, что разработка и оптимизация чат-ботов представляют собой важное направление для современных бизнес-процессов. Чат-боты являются мощным инструментом автоматизации и оптимизации бизнес-взаимодействия с пользователями, и их актуальность постоянно растет.

Исследование показало, что для создания успешных чат-ботов, сочетающих в себе функциональность, легкость в использовании и высокий уровень понимания пользователей, необходимо комплексное исследование и разработка. Это включает в себя работу не только программистов, но и специалистов по дизайну диалоговых ассистентов, UX/UI дизайнеров, бизнес-аналитиков и других экспертов.

Для достижения оптимальных результатов в создании чат-ботов необходимо уделить внимание не только технической реализации, но и пониманию потребностей и ожиданий пользователей. Этот процесс требует интегрированного подхода, где каждый участник команды вносит свой вклад в создание более эффективных и удовлетворяющих решений.

Исследование показало, что оптимальный баланс между функциональностью и понятностью пользователю может быть достигнут только при сотрудничестве специалистов из разных областей. Целая команда, включающая в себя программистов, дизайнеров диалоговых ассистентов, UX/UI дизайнеров, бизнес-аналитиков и других экспертов, способна создавать чат-боты, которые не только автоматизируют бизнес-процессы, но и обеспечивают легкость и комфорт взаимодействия для пользователей.

В итоге, данное исследование подтвердило значимость чат-ботов в современном мире бизнеса и подчеркнуло важность интеграции различных специалистов для создания ботов, которые будут наиболее эффективными и удовлетворяющими запросы пользователей.

\bibliographystyle{alpha}
\bibliography{sample}

\begin{thebibliography} {50}
\bibitem{1} George D. Greenwade. The Comprehensive Tex Archive Network (CTAN). TUGBoat, 14(3):342–351, 1993.

\bibitem{2}
B.M. Glinskiy, I.M. Kulikov, I.G. Chernykh, A.V. Snytnikov, A.V. Sapetina, D.V. Weins.  The Integrated Approach to Solving Large-Size Physical Problems on Supercomputers.Принято к публикации в Communications in Computer and Information Science, изд-во Springer, 2017 г.;

\bibitem{3} 
Библиотека Python "os" - Разные интерфейсы операционной системы [Электронный ресурс]. Режим доступа:  Свободный. URL: "https://translated.turbopages.org/proxy_u/en-ru.ru.4b04840d-651949ed-f730bcf4-74722d776562/https/docs.python.org/3.8/library/os.html";

\bibitem{4}
Как сделать из Python-скрипта исполняемый файл [Электронный ресурс] Режим доступа: Свободный. URL: "https://habr.com/ru/companies/slurm/articles/746622/"

\bibitem{5}
С.А. Белоусова, Ю.И. Рогозов
АНАЛИЗ ПОДХОДОВ К СОЗДАНИЮ ПОЛЬЗОВАТЕЛЬСКОГО ИНТЕРФЕЙСА. Режим доступа: Свободный. URL: "https://cyberleninka.ru/article/n/analiz-podhodov-k-sozdaniyu-polzovatelskogo-interfeysa"

\bibitem{6}
УПРАВЛЕНИЕ ПРОЕКТИРОВАНИЕМ И РЕАЛИЗАЦИЕЙ ПОЛЬЗОВАТЕЛЬСКПГН ИНТЕРФЕЙСА НА 
ОСНОВЕ ОНТОЛОГИЙ. В. В. Грибова, А. С. Клещев. Режим доступа: Свободный. URL: "https://cyberleninka.ru/article/n/upravlenie-proektirovaniem-i-realizatsiey-
polzovatelskogo-interfeysa-na-osnove-ontologiy"

\bibitem{7}
Н.В. НОВОЖИЛОВА, ОСОБЕННОСТИ ПРОЕКТИРОВАНИЯ ДРУЖЕСТВЕННЫХ ИНТЕРФЕЙСОВ ДЛЯ ПОЛЬЗОВАТЕЛЕЙ-
ЭКОНОМИСТОВ. Режим доступа: Свободный. URL: "https://cyberleninka.ru/article/n/osobennosti-proektirovaniya-druzhestvennyh-interfeysov-dlya-polzovateley-ekonomistov"

\bibitem{8}
КАТЕГОРИЯ «ЕСТЕСТВЕННОСТЬ» В КЛАССИФИКАЦИИ ПОЛЬЗОВАТЕЛЬСКИХ ИНТЕРФЕЙСОВ, Н.Н. Зильберман, С.А. Алексеев. Режим доступа: Свободный. URL: "https://cyberleninka.ru/article/n/kategoriya-estestvennost-v-klassifikatsii-polzovatelskih-interfeysov"

\bibitem{9}
УНИФИЦИРОВАННАЯ МОДЕЛЬ ПОЛЬЗОВАТЕЛЬСКИХ ИНТЕРФЕЙСОВ ИНТЕЛЛЕКТУАЛЬНЫХ СИСТЕМ, Д.Г. КОЛБ. Режим доступа: Свободный. URL: "https://cyberleninka.ru/article/n/unifitsirovannaya-model-polzovatelskih-interfeysov-intellektualnyh-sistem"

\bibitem{10}
 ИСПОЛЬЗОВАНИЕ СБОРОЧНОЙ ТЕХНОЛОГИИ ДЛЯ ПОСТРОЕНИЯ ПОЛЬЗОВАТЕЛЬСКИХ ИНТЕРФЕЙСОВ СЕТЕВОЙ ИНФОРМАЦИОННО-ВЫЧИСЛИТЕЛЬНОЙ СИСТЕМЫ, С. В. Пискунов, С. В. Кратов, М. Б. Остапкевич, А. В. Веселов. Режим доступа: Свободный. URL: "https://cyberleninka.ru/article/n/ispolzovanie-sborochnoy-tehnologii-dlya-postroeniya-polzovatelskih-interfeysov-
setevoy-informatsionno-vychislitelnoy-sistemy"

\bibitem{11}
ЕСТЕСТВЕННО-ЯЗЫКОВЫЕ ИНТЕРФЕЙСЫ ИНТЕЛЛЕКТУАЛЬНЫХ ВОПРОСНО-ОТВЕТНЫХ СИСТЕМ, В.А. ЖИТКО, В Н. ВЯЛЬЦЕВ, Ю.С. ГЕЦЕВИЧ, А.А. КУЗЬМИН3. Режим доступа: Свободный. URL:  "https://cyberleninka.ru/article/n/estestvenno-yazykovye-interfeysy-intellektualnyh-voprosno-otvetnyh-sistem"

\bibitem{12}
ЕСТЕСТВЕННО-ЯЗЫКОВОЙ ПОЛЬЗОВАТЕЛЬСКИЙ ИНТЕРФЕЙС ДИАЛОГОВОЙ СИСТЕМЫ, Р.В. Посевкин, И.А. Бессмертный. Режим доступа: Свободный. URL: "https://cyberleninka.ru/article/n/estestvenno-yazykovoy-polzovatelskiy-interfeys-dialogovoy-sistemy"

\bibitem{13}
Л.Е. Малыгина ЧАТ-БОТЫ И ИСКУССТВЕННЫЙ ИНТЕЛЛЕКТ: ПЕРСПЕКТИВЫ РАЗВИТИЯ ТЕЛЕВИЗИОННОГО 
ПРОМОДИСКУРСА. Режим доступа: Свободный. URL: "https://cyberleninka.ru/article/n/chat-boty-i-iskusstvennyy-intellekt-perspektivy-razvitiya-televizionnogo-promodiskursa"

\bibitem{14}
The Role of BAs in User Interface Design//Business analyst learnings [Электронный ресурс]. Режим доступа: Свободный. URL: "https://businessanalystlearnings.com/blog/2014/7/29/the-role-of-basin-user-interface-design"

\bibitem{15}
Вендров А.М. Проектирование программного обеспечения экономических информационных систем. М. : Финансы и статистика, 2005. 544 с.

\bibitem{16}
Алан Купер об интерфейсе. Основы проектирования взаи модействия / пер. с англ. СПб. : Символ-Плюс, 2009. 688 с.

\bibitem{17}
Андреев В. О чем надо помнить при разработке пользовательского интерфейса [Электронный ресурс]//Usability в России. Режим доступа: Свободный. URL: "http://www.usability.ru/Articles/instruction.html"

\bibitem{18}
Раскин Дж. Интерфейс: новые направления в проектировании компьютерных систем. М. : Символ-Плюс, 2005. 160 с.

\bibitem{19}
Андрейчиков А. В., Андрейчикова О. Н. Интеллектуаль ные информационные системы. М. : Финансы и статистика, 2004. 422 с.

\bibitem{20}
Головач В. Дизайн пользовательского интерфейса. Искусство мыть слона [Электронный ресурс] //Юзетикс. Режим доступа: Свободный. URL: "http://www.usethics.ru/lib"

\bibitem{21}
Торес Р. Дж. Практическое руководство по проектированию и разработке пользовательского интерфейса / пер. с англ. М. : Вильямс, 2002. 400 с.

\bibitem{22}
Жарков С. Shareware: профессиональная разработка и про движение программ. СПб. : БХВПетербург, 2002. 320 c.

\bibitem{23}
Руководство по WPF [Электронный ресурс] //Сайт о программировании. Режим доступа: Свободный. URL: "https://metanit.com/sharp/wpf"

\bibitem{24}
Протокол заседания Совета по стратегическому развитию и приоритетным проектам «О программе «Цифровая экономика» от 5 июля 2017 г. Режим доступа: Свободный. URL: "http://www.kremlin.ru/acts/assignments/orders/55100"

\bibitem{25}
Цифровая Россия: новая реальность (2017) / McKinsey. Режим доступа: Свободный. URL:
"http://www.tadviser.ru/images/c/c2/ Digital-Russia-report.pdf"

\bibitem{26}
В чем разница между искусственным интеллектом, машинным обучением и глубоким обучением? (2016) / Nvidia. Режим доступа: Свободный. URL: "http://www.nvidia.ru/ 
object/whats-difference-ai-machine-learning-deeplearning-blog-ru.html"

\bibitem{27}
Искусственный интеллект (ИИ) / Artificial Intelligence (AI) как ключевой 
фактор цифровизации глобальной экономики (2017) / CRN/RE. Режим доступа: Свободный. URL: "https://www.crn.ru/news/detail.php? ID=117544"

\bibitem{28}
Мировой рынок систем и технологий искусственного интеллекта: становление и тенденции развития, V. Matyushok, V. Krasavina. Режим доступа: Свободный. URL: 
"https://www.researchgate.net/publication/345085060_Mirovoj_rynok_sistem_i_tehnologij_iskusstvennogo_intellekta_stanovlenie_i_tendencii_razvitia"

\bibitem{29}
Искусственный интеллект как стратегический инструмент экономического развития страны и совершенствования ее государственного управления. Часть 1. Опыт Великобритании и сша, И.А Соколов, В.И. Дрожжинов, А.Н. Райков, В.П. Куприяновский, Д.Е. Намиот. Режим доступа: Свободный. URL: "https://cyberleninka.ru/article/n/iskusstvennyy-intellekt-kak-strategicheskiy-instrument-ekonomicheskogo-razvitiya-strany-i-sovershenstvovaniya-ee-gosudarstvennogo"

\bibitem{30}
С.А. Шилова, А.А. Крючкова ЛИНГВОДИДАКТИЧЕСКИЙ ПОТЕНЦИАЛ ЧАТ-БОТОВ // Иностранные языки в контексте межкультурной коммуникации. 2021. №XIII. Режим доступа: Свободный. URL: "https://cyberleninka.ru/article/n/lingvodidakticheskiy-potentsial-chat-botov" (дата обращения: 17.12.2023).

\bibitem{31}
Декалюк А.О. СФЕРЫ ИСПОЛЬЗОВАНИЯ ЧАТ-БОТОВ // Теория и практика современной науки. 2021. №6 (72). Режим доступа: Свободный. URL: "https://cyberleninka.ru/article/n/sfery-ispolzovaniya-chat-botov" (дата обращения: 17.12.2023).

\bibitem{32}
Лысенко Александр Вячеславович Технологии чат-ботов в crm-технологиях // КНЖ. 2019. №1 (26). Режим доступа: Свободный. URL: "https://cyberleninka.ru/article/n/tehnologii-chat-botov-v-crm-tehnologiyah" (дата обращения: 17.12.2023).

\bibitem{33}
Малекова Виктория Алексеевна, Романова Екатерина Владимировна ПЕРСПЕКТИВЫ ИСПОЛЬЗОВАНИЯ ЧАТ-БОТОВ В HR-СЕКТОРЕ // Сервис в России и за рубежом. 2023. №1 (103). Режим доступа: Свободный. URL: "https://cyberleninka.ru/article/n/perspektivy-ispolzovaniya-chat-botov-v-hr-sektore" (дата обращения: 17.12.2023).

\bibitem{34}
Голованова Ольга Станиславовна, Масюк Наталья Николаевна ЧАТ-БОТ - DIGITAL ПОМОЩНИК РЕКРУТЕРА // Территория новых возможностей. 2020. №4. Режим доступа: Свободный. URL: "https://cyberleninka.ru/article/n/chat-bot-digital-pomoschnik-rekrutera" (дата обращения: 17.12.2023).

\bibitem{35}
Гасангулиева Матанат Мухаммад, Байрамова Вусала Эльмар ТЕХНОЛОГИЯ ИСКУССТВЕННОГО ИНТЕЛЛЕКТА, ИСПОЛЬЗУЕМАЯ В ЦИФРОВОМ УПРАВЛЕНИИ: ЧАТ-БОТ // ELS. 2023. №ноябрь. Режим доступа: Свободный. URL: "https://cyberleninka.ru/article/n/tehnologiya-iskusstvennogo-intellekta-ispolzuemaya-v-tsifrovom-upravlenii-chat-bot" (дата обращения: 17.12.2023).

\bibitem{36}
Слепцова Юлия Николаевна Автоматизация маркетинговых процессов при помощи чат-бота // Научный журнал. 2020. №3 (48). Режим доступа: Свободный. URL: "https://cyberleninka.ru/article/n/avtomatizatsiya-marketingovyh-protsessov-pri-pomoschi-chat-bota" (дата обращения: 17.12.2023).

\bibitem{37}
Ольга Сергеевна Биккулова, Марина Игоревна Ивкина ЧАТ-БОТ В МЕТОДИКЕ ПРЕПОДАВАНИЯ РКИ // МИРС. 2021. №1. Режим доступа: Свободный. URL: "https://cyberleninka.ru/article/n/chat-bot-v-metodike-prepodavaniya-rki" (дата обращения: 17.12.2023).

\bibitem{38}
Матвеева Н.Ю., Золотарюк А.В. Технологии создания и применения чат-ботов // Научные записки молодых исследователей. 2018. №1. Режим доступа: Свободный. URL: "https://cyberleninka.ru/article/n/tehnologii-sozdaniya-i-primeneniya-chat-botov" (дата обращения: 17.12.2023).

\bibitem{39}
Ураев Д. А. Классификация и методы создания чат-бот приложений // International scientific review. 2019. №LXIV. Режим доступа: Свободный. URL: "https://cyberleninka.ru/article/n/klassifikatsiya-i-metody-sozdaniya-chat-bot-prilozheniy" (дата обращения: 17.12.2023).

\bibitem{40}
Половнева М.В. АНАЛИЗ РАЗВИТИЯ И ПРИМЕНЕНИЯ ТЕХНОЛОГИИ ЧАТ-БОТ // Теория и практика современной науки. 2018. №6 (36). Режим доступа: Свободный. URL: "https://cyberleninka.ru/article/n/analiz-razvitiya-i-primeneniya-tehnologii-chat-bot" (дата обращения: 17.12.2023).

\bibitem{41}
Махорт Анна Константиновна, Синенко Полина Дмитриевна TELEGRAM КАК УНИКАЛЬНАЯ ПЛОЩАДКА ИНДУСТРИИ МЕССЕНДЖЕРОВ // Огарёв-Online. 2021. №2 (155). Режим доступа: Свободный. URL: "https://cyberleninka.ru/article/n/telegram-kak-unikalnaya-ploschadka-industrii-messendzherov" (дата обращения: 17.12.2023).

\bibitem{42}
Могилевская Н. С., Самойленко Г. П. ЧАТ-БОТ ДЛЯ ТЕСТИРОВАНИЯ ЗНАНИЙ НА ПЛАТФОРМЕ МЕССЕНДЖЕРА TELEGRAM // Молодой исследователь Дона. 2022. №3 (36). URL: https://cyberleninka.ru/article/n/chat-bot-dlya-testirovaniya-znaniy-na-platforme-messendzhera-telegram (дата обращения: 17.12.2023).

\bibitem{43}
Екатерина Петровна Турбина Дистанционные технологии в преподавании иностранных языков (на примере приложения telegram) // Вестник Шадринского государственного педагогического университета. 2023. №1 (57). URL: https://cyberleninka.ru/article/n/distantsionnye-tehnologii-v-prepodavanii-inostrannyh-yazykov-na-primere-prilozheniya-telegram (дата обращения: 17.12.2023).

\bibitem{44}
Валинурова Анна Александровна, Балабанова Наталья Владимировна, Маценков Иван Алексеевич АЛГОРИТМ РАЗРАБОТКИ TELEGRAM -БОТА - ПРОДУКТИВНОГО ПОМОЩНИКА СОВРЕМЕННОГО БИЗНЕСА // Современные наукоемкие технологии. Региональное приложение . 2023. №2 (74). URL: https://cyberleninka.ru/article/n/algoritm-razrabotki-telegram-bota-produktivnogo-pomoschnika-sovremennogo-biznesa (дата обращения: 17.12.2023).

\bibitem{45}
Шумилина Мария Александровна, Коробко Анна Владимировна РАЗРАБОТКА ЧАТ-БОТА НА ЯЗЫКЕ ПРОГРАММИРОВАНИЯ PYTHON В МЕССЕНДЖЕРЕ "TELEGRAM" // Научные известия. 2022. №28. URL: https://cyberleninka.ru/article/n/razrabotka-chat-bota-na-yazyke-programmirovaniya-python-v-messendzhere-telegram (дата обращения: 17.12.2023).

\bibitem{46}
Белых Ольга Александровна ОБРАЗОВАТЕЛЬНЫЙ КОНТЕНТ МЕССЕНДЖЕРА «TELEGRAM» КАК ИНСТРУМЕНТ ПОВЫШЕНИЯ РЕЗУЛЬТАТИВНОСТИ ОБРАЗОВАТЕЛЬНОГО ПРОЦЕССА // Педагогический ИМИДЖ. 2021. №3 (52). URL: https://cyberleninka.ru/article/n/obrazovatelnyy-kontent-messendzhera-telegram-kak-instrument-povysheniya-rezultativnosti-obrazovatelnogo-protsessa (дата обращения: 17.12.2023).

\bibitem{47}
Верещагина Е.А., Рудниченко А.К., Рудниченко Д.С. МОНИТОРИНГ СТАБИЛЬНОСТИ СЕТЕВОЙ ИНФРАСТРУКТУРЫ С ИСПОЛЬЗОВАНИЕМ МЕССЕНДЖЕРА TELEGRAM // ИВД. 2020. №11 (71). URL: https://cyberleninka.ru/article/n/monitoring-stabilnosti-setevoy-infrastruktury-s-ispolzovaniem-messendzhera-telegram (дата обращения: 17.12.2023).

\bibitem{48}
Сунцова Д. И., Павлов В. А., Макаренко З. В., Бахолдин П. П., Полицинский А. С., Кремлев А. С., Маргун А. А. РАЗРАБОТКА TELEGRAM-БОТА «ОПРЕДЕЛЕНИЕ УРОВНЯ ГОТОВНОСТИ ТЕХНОЛОГИИ» // Экономика науки. 2022. №1. URL: https://cyberleninka.ru/article/n/razrabotka-telegram-bota-opredelenie-urovnya-gotovnosti-tehnologii (дата обращения: 17.12.2023).

\bibitem{49}
Ro’Ziqulova, Maqsuda Abriy Qizi INTELLEKTUAL TEST SINOVLARINI O‘TKAZUVCHI TELEGRAM BOT YARATISH // ORIENSS. 2022. №6. URL: https://cyberleninka.ru/article/n/intellektual-test-sinovlarini-o-tkazuvchi-telegram-bot-yaratish (дата обращения: 17.12.2023).

\bibitem{50}
Филатова Зульфия Мирсайжановна, Закирова Нурия Ришатовна СОЗДАНИЕ TELEGRAM-БОТА ДЛЯ АВТОМАТИЗАЦИИ АДМИНИСТРАТИВНОЙ ДЕЯТЕЛЬНОСТИ // Проблемы современного педагогического образования. 2023. №79-4. URL: https://cyberleninka.ru/article/n/sozdanie-telegram-bota-dlya-avtomatizatsii-administrativnoy-deyatelnosti (дата обращения: 17.12.2023).

\bibitem{51}
Краснова Марина Николаевна ИСПОЛЬЗОВАНИЕ КАНАЛА TELEGRAM ПРИ ПОДГОТОВКЕ УЧАЩИХСЯ К ЭКЗАМЕНАМ // Вестник науки и образования. 2022. №7-1 (127). URL: https://cyberleninka.ru/article/n/ispolzovanie-kanala-telegram-pri-podgotovke-uchaschihsya-k-ekzamenam (дата обращения: 17.12.2023).

\bibitem{52}
Коноплев Дмитрий Эдуардович Telegram как новая среда коммуникации в СМИ и соцсетях // Знак: проблемное поле медиаобразования. 2017. №3 (25). URL: https://cyberleninka.ru/article/n/telegram-kak-novaya-sreda-kommunikatsii-v-smi-i-sotssetyah (дата обращения: 17.12.2023).

\bibitem{53}
Лытнева Анна Андреевна, Дубинина Анна Эмировна TELEGRAM КАК НОВАЯ ПЛОЩАДКА В СРЕДСТВАХ МАССОВОЙ КОММУНИКАЦИИ // Материалы Афанасьевских чтений. 2020. №2 (31). URL: https://cyberleninka.ru/article/n/telegram-kak-novaya-ploschadka-v-sredstvah-massovoy-kommunikatsii (дата обращения: 17.12.2023).

\bibitem{54}
Баранова Екатерина Андреевна, Андрианова Дарья Дмитриевна СОВРЕМЕННЫЙ РОССИЙСКИЙ РЫНОК ДЕЛОВОЙ ИНФОРМАЦИИ В УСЛОВИЯХ КОНКУРЕНЦИИ ЕЖЕДНЕВНЫХ СМИ И TELEGRAM-КАНАЛОВ // Litera. 2022. №8. URL: https://cyberleninka.ru/article/n/sovremennyy-rossiyskiy-rynok-delovoy-informatsii-v-usloviyah-konkurentsii-ezhednevnyh-smi-i-telegram-kanalov (дата обращения: 17.12.2023).

\bibitem{55}
Дементьева К. В. РАЗВИТИЕ TELEGRAM-КАНАЛОВ В МЕДИАПРОСТРАНСТВЕ РОССИЙСКИХ РЕГИОНОВ: СПЕЦИФИКА, ТИПОЛОГИЯ, ПЕРСПЕКТИВЫ РАЗВИТИЯ (НА ПРИМЕРЕ TELEGRAM-КАНАЛОВ РЕСПУБЛИКИ МОРДОВИЯ) // Вестник НГУ. Серия: История, филология. 2021. №6. URL: https://cyberleninka.ru/article/n/razvitie-telegram-kanalov-v-mediaprostranstve-rossiyskih-regionov-spetsifika-tipologiya-perspektivy-razvitiya-na-primere-telegram (дата обращения: 17.12.2023).

\bibitem{56}
Абрамова А.И. ИСПОЛЬЗОВАНИЕ TELEGRAM-БОТА В ОБРАЗОВАТЕЛЬНОМ ПРОЦЕССЕ ВУЗА // Вестник науки. 2022. №1 (46). URL: https://cyberleninka.ru/article/n/ispolzovanie-telegram-bota-v-obrazovatelnom-protsesse-vuza (дата обращения: 17.12.2023).

\bibitem{57}
Андриянова Софья Сархановна, Веретено Александра Александровна Использование мессенджера Telegram для продвижения бренда // Economics. 2018. №3 (35). URL: https://cyberleninka.ru/article/n/ispolzovanie-messendzhera-telegram-dlya-prodvizheniya-brenda (дата обращения: 17.12.2023).

\bibitem{58}
Алексеев В.А., Домашнев П.А., Лаврухина Т.В., Назаркин О.А. Использование функций мессенджера Telegram для обмена сообщениями между узлами распределенной вычислительной системы // International Journal of Open Information Technologies. 2019. №6. URL: https://cyberleninka.ru/article/n/ispolzovanie-funktsiy-messendzhera-telegram-dlya-obmena-soobscheniyami-mezhdu-uzlami-raspredelennoy-vychislitelnoy-sistemy (дата обращения: 17.12.2023).

\bibitem{59}
Шакиров Р.И., Татаурова А.С. Автоматизация учебного расписания через Telegram-bot // Символ науки. 2020. №1. URL: https://cyberleninka.ru/article/n/avtomatizatsiya-uchebnogo-raspisaniya-cherez-telegram-bot (дата обращения: 17.12.2023).

\bibitem{60}
Байкова К.Д., Медведева T.A. АНАЛИЗ И РАЗРАБОТКА ФУНКЦИОНАЛЬНОГО TELEGRAM-БОТА // Молодой исследователь Дона. 2021. №6 (33). URL: https://cyberleninka.ru/article/n/analiz-i-razrabotka-funktsionalnogo-telegram-bota (дата обращения: 17.12.2023).

\bibitem{61}
Бийбосунов Б. И., Бийбосунова С. К., Жолочубеков Н. Ж. Описание концепции Telegram ботов и их разработка // Colloquium-journal. 2020. №7 (59). URL: https://cyberleninka.ru/article/n/opisanie-kontseptsii-telegram-botov-i-ih-razrabotka (дата обращения: 17.12.2023).

\bibitem{62}
Назаров Алишер Искендерович, Мухаммадиев Нурфайз РАЗРАБОТКА МОДИФИЦИРОВАННОГО TELEGRAM-БОТА // SAI. 2023. №Special Issue 3. URL: https://cyberleninka.ru/article/n/razrabotka-modifitsirovannogo-telegram-bota (дата обращения: 17.12.2023).

\bibitem{63}
Артамонова Мария Валериевна, Мамбетов Акбар Азаматович, Тулина Екатерина Валерьевна Чат-бот как инструмент в работе переводчика // Litera. 2023. №8. URL: https://cyberleninka.ru/article/n/chat-bot-kak-instrument-v-rabote-perevodchika (дата обращения: 17.12.2023).

\bibitem{64}
Т.А. Дейнеко, Д.А. Бобров ЧАТ-БОТ ВКОНТАКТЕ «РАСПИСАНИЕ ЗАНЯТИЙ ОМГУ» // МСиМ. 2020. №3 (55). URL: https://cyberleninka.ru/article/n/chat-bot-vkontakte-raspisanie-zanyatiy-omgu (дата обращения: 17.12.2023).

\bibitem{65}
Параскевов Александр Владимирович, Каденцева Анастасия Александровна, Мороз Сергей Игоревич Перспективы и особенности разработки чат-ботов // Научный журнал КубГАУ. 2017. №130. URL: https://cyberleninka.ru/article/n/perspektivy-i-osobennosti-razrabotki-chat-botov (дата обращения: 17.12.2023).

\bibitem{66}
Бахтин И.В. РАЗРАБОТКА ЧАТ-БОТОВ ДЛЯ АВТОМАТИЗАЦИИ БИЗНЕС- ПРОЦЕССОВ // Форум молодых ученых. 2020. №2 (42). URL: https://cyberleninka.ru/article/n/razrabotka-chat-botov-dlya-avtomatizatsii-biznes-protsessov (дата обращения: 17.12.2023).

\bibitem{67}
Дулёв А.А. ИННОВАЦИОННЫЕ БАНКОВСКИЕ ПРОДУКТЫ И ИХ РАЗВИТИЕ В РОССИЙСКОЙ ФЕДЕРАЦИИ // Хроноэкономика. 2020. №5 (26). URL: https://cyberleninka.ru/article/n/innovatsionnye-bankovskie-produkty-i-ih-razvitie-v-rossiyskoy-federatsii (дата обращения: 17.12.2023).

\bibitem{68}
Строев Владимир Витальевич, Тихонов Алексей Иванович АНАЛИЗ ИСПОЛЬЗОВАНИЯ ЧАТ-БОТА КАК ИНСТРУМЕНТА ОНЛАЙН-ОБУЧЕНИЯ ПЕРСОНАЛА // Московский экономический журнал. 2022. №6. URL: https://cyberleninka.ru/article/n/analiz-ispolzovaniya-chat-bota-kak-instrumenta-onlayn-obucheniya-personala (дата обращения: 17.12.2023).

\bibitem{69}
Яндыбаева Н.В., Акельев И.В. РАЗРАБОТКА ЧАТ-БОТА ДЛЯ ТЕХНИЧЕСКОЙ ПОДДЕРЖКИ ПОЛЬЗОВАТЕЛЕЙ В МЕДИЦИНСКОМ УЧРЕЖДЕНИИ // Международный журнал гуманитарных и естественных наук. 2022. №9-1. URL: https://cyberleninka.ru/article/n/razrabotka-chat-bota-dlya-tehnicheskoy-podderzhki-polzovateley-v-meditsinskom-uchrezhdenii (дата обращения: 17.12.2023).

\bibitem{70}
Минина Вера Николаевна HR-боты в управлении человеческими ресурсами организации // Вестник Санкт-Петербургского университета. Менеджмент. 2019. №3. URL: https://cyberleninka.ru/article/n/hr-boty-v-upravlenii-chelovecheskimi-resursami-organizatsii (дата обращения: 17.12.2023).

\bibitem{71}
Горячкин Б.С., Галичий Д.А., Цапий В.С., Бурашников В.В., Крутов Т.Ю. ЭФФЕКТИВНОСТЬ ИСПОЛЬЗОВАНИЯ ЧАТ-БОТОВ В ОБРАЗОВАТЕЛЬНОМ ПРОЦЕССЕ // E-Scio. 2021. №4 (55). URL: https://cyberleninka.ru/article/n/effektivnost-ispolzovaniya-chat-botov-v-obrazovatelnom-protsesse (дата обращения: 17.12.2023).

\bibitem{72}
Ураев Д. А. Классификация и методы создания чат-бот приложений // International scientific review. 2019. №LXIV. URL: https://cyberleninka.ru/article/n/klassifikatsiya-i-metody-sozdaniya-chat-bot-prilozheniy (дата обращения: 17.12.2023).

\bibitem{73}
Сахаров В.И. СРАВНИТЕЛЬНЫЙ АНАЛИЗ СУЩЕСТВУЮЩИХ БОТ ПЛАТФОРМ // Теория и практика современной науки. 2023. №1 (91). URL: https://cyberleninka.ru/article/n/sravnitelnyy-analiz-suschestvuyuschih-bot-platform (дата обращения: 17.12.2023).

\bibitem{74}
Лебедева Татьяна Евгеньевна, Лазутина Антонина Леонардовна, Царьков Николай Романович, Башкаева Ольга Анатольевна Digital-технологии в HR // Московский экономический журнал. 2018. №5 (2). URL: https://cyberleninka.ru/article/n/digital-tehnologii-v-hr (дата обращения: 17.12.2023).

\bibitem{75}
Гладкая Ксения Викторовна, Тихонов Алексей Иванович Автоматизация задач в области управления персоналом с помощью HR-бота // Московский экономический журнал. 2020. №4. URL: https://cyberleninka.ru/article/n/avtomatizatsiya-zadach-v-oblasti-upravleniya-personalom-s-pomoschyu-hr-bota (дата обращения: 17.12.2023).

\bibitem{76}
Фурсова Т.В., Терновская Т.Н., Романов Е.В. ВНЕДРЕНИЕ ЧАТ-БОТОВ В ТЕХНОЛОГИИ ДИСТАНЦИОННОГО БАНКОВСКОГО ОБСЛУЖИВАНИЯ // Вестник МФЮА. 2020. №4. URL: https://cyberleninka.ru/article/n/vnedrenie-chat-botov-v-tehnologii-distantsionnogo-bankovskogo-obsluzhivaniya (дата обращения: 17.12.2023).

\bibitem{77}
Валевич А.Г., Кудряшов В.С., Фомина В.В. ЧАТ-БОТЫ КАК ЭЛЕМЕНТ СИСТЕМЫ УПРАВЛЕНИЯ ОРГАНИЗАЦИЕЙ // Ученые записки Тамбовского отделения РоСМУ. 2023. №29. URL: https://cyberleninka.ru/article/n/chat-boty-kak-element-sistemy-upravleniya-organizatsiey (дата обращения: 17.12.2023).

\bibitem{78}
Игамбердиева К.Э. ИСПОЛЬЗОВАНИЕ ИСКУССТВЕННОГО ИНТЕЛЛЕКТА В РАЗВИТИИ ЭКОНОМИКИ ПРЕДПРИЯТИЯ // Экономика и социум. 2023. №5-1 (108). URL: https://cyberleninka.ru/article/n/ispolzovanie-iskusstvennogo-intellekta-v-razvitii-ekonomiki-predpriyatiya (дата обращения: 17.12.2023).

\bibitem{79}
Князева В.И., Городецкая О.Ю. ИСПОЛЬЗОВАНИЕ ЧАТ-БОТА В КАЧЕСТВЕ ИНФОРМАЦИОННОЙ СИСТЕМЫ РЕСТОРАННОГО БИЗНЕСА // Хроноэкономика. 2020. №4 (25). URL: https://cyberleninka.ru/article/n/ispolzovanie-chat-bota-v-kachestve-informatsionnoy-sistemy-restorannogo-biznesa (дата обращения: 17.12.2023).

\bibitem{80}
Николаева Галина Леонидовна, Тухбатов Расиль Равилович ЦИФРОВЫЕ ПОМОЩНИКИ КАК ИНСТРУМЕНТ ОПТИМИЗАЦИИ ВЗАИМОДЕЙСТВИЯ ЭКОНОМИЧЕСКИХ АГЕНТОВ // Современные информационные технологии и ИТ-образование. 2020. №1. URL: https://cyberleninka.ru/article/n/tsifrovye-pomoschniki-kak-instrument-optimizatsii-vzaimodeystviya-ekonomicheskih-agentov (дата обращения: 17.12.2023).

\bibitem{81}
Куликова О.М., Суворова С.Д. РОЛЬ ЧАТ-БОТОВ В ПОСТРОЕНИИ ЭФФЕКТИВНЫХ КОММУНИКАЦИЙ // Международный журнал гуманитарных и естественных наук. 2021. №4-3. URL: https://cyberleninka.ru/article/n/rol-chat-botov-v-postroenii-effektivnyh-kommunikatsiy (дата обращения: 17.12.2023).

\bibitem{82}
Ураев Денис Алексеевич Метрики для оценки качества чат-бот приложений // Наука, техника и образование. 2019. №9 (62). URL: https://cyberleninka.ru/article/n/metriki-dlya-otsenki-kachestva-chat-bot-prilozheniy (дата обращения: 17.12.2023).

\bibitem{83}
Махина Екатерина Дмитриевна, Пальчунов Дмитрий Евгеньевич Программная система для определения речевых действий в текстах естественного языка // Вестник НГУ. Серия: Информационные технологии. 2018. №4. URL: https://cyberleninka.ru/article/n/programmnaya-sistema-dlya-opredeleniya-rechevyh-deystviy-v-tekstah-estestvennogo-yazyka (дата обращения: 17.12.2023).

\bibitem{84}
Гольчевский Ю.В., Непеин А.В. ПРОЕКТИРОВАНИЕ И РАЗРАБОТКА ЧАТ-БОТА ДЛЯ ПРЕДСТАВЛЕНИЯ РАСПИСАНИЯ В СОЦИАЛЬНОЙ СЕТИ // Вестник Сыктывкарского университета. Серия 1. Математика. Механика. Информатика. 2021. №3 (40). URL: https://cyberleninka.ru/article/n/proektirovanie-i-razrabotka-chat-bota-dlya-predstavleniya-raspisaniya-v-sotsialnoy-seti (дата обращения: 17.12.2023).

\bibitem{85}
"Designing with the Mind in Mind: Simple Guide to Understanding User Interface Design Rules" by Jeff Johnson. URL: "https://lib.fbtuit.uz/assets/files/Designing-withtheMindinMindSimple-Johnson-Kaufmann2010.pdf"

\bibitem{86}
"Don't Make Me Think, Revisited: A Common Sense Approach to Web Usability" by Steve Krug. URL:
"https://pdflake.com/wp-content/uploads/2022/09/Dont-Make-Me-Think-PDF.pdf"

\bibitem{87}
"The Design of Everyday Things" by Don Norman. URL:
"https://streettrotter.com/wp-content/uploads/2020/08/The-Design-of-Everyday-Things-Book-by-Don-Norman.pdf"

\bibitem{88}
"Measuring the User Experience: Collecting, Analyzing, and Presenting Usability Metrics" by William Albert. URL: "https://medium.com/@GemmaAntho98100/epub-measuring-the-user-experience-collecting-analyzing-and-presenting-usability-metrics-f923d548dc"

\bibitem{89}
"Seductive Interaction Design: Creating Playful, Fun, and Effective User Experiences" by Stephen Anderson. URL: "https://www.pdfdrive.com/seductive-interaction-design-creating-playful-fun-and-effective-user-experiences-voices-that-matter-e193187842.html"

\bibitem{90}
"Hooked: How to Build Habit-Forming Products" by Nir Eyal. URL: "https://medium.com/@nansyonlinebusinnes/book-review-hooked-how-to-build-habit-forming-products-by-nir-eyal-1a2c4deeb43a"

\bibitem{91}
"Usability Engineering" by Jakob Nielsen. URL: "https://drive.google.com/file/d/0B5tR1YhNBlD2Wlg5NW1fS09sOFU/view?usp=sharing"

\bibitem{92}
"Lean UX: Designing Great Products with Agile Teams" by Jeff Gothelf. URL:
"https://www.pdfdrive.com/lean-ux-designing-great-products-with-agile-teams-e185409456.html"

\bibitem{93}
"The Elements of User Experience: User-Centered Design for the Web and Beyond" by Jesse James Garrett. URL: "https://www.pdfdrive.com/the-elements-of-user-experience-user-centered-design-for-the-web-and-beyond-2nd-edition-voices-that-matter-e157223396.html"

\bibitem{94}
"100 Things Every Designer Needs to Know About People" by Susan Weinschenk. URL: "https://www.pdfdrive.com/100-things-every-designer-needs-to-know-about-people-what-makes-them-tick-e156711199.html"

\bibitem{95}
"User Story Mapping: Discover the Whole Story, Build the Right Product" by Jeff Patton. URL: https://www.pdfdrive.com/user-story-mapping-discover-the-whole-story-build-the-right-product-e157789213.html

\bibitem{96}
"Designing for Interaction: Creating Innovative Applications and Devices" by Dan Saffer. URL: "https://books.google.ru/books?id=Dd3Hcs9jeoUC&hl=ru"

\bibitem{97}
"About Face: The Essentials of Interaction Design" by Alan Cooper. URL: "https://medium.com/@6cuongseven9899g/about-face-the-essentials-of-interaction-design-full-acces-c0ddd099e9a1"

\bibitem{98}
"Information Architecture: For the Web and Beyond" by Louis Rosenfeld and Peter Morville. URL: "https://www.pdfdrive.com/information-architecture-for-the-web-and-beyond-e158738770.html"

\bibitem{99}
"The Lean Startup: How Today's Entrepreneurs Use Continuous Innovation to Create Radically Successful Businesses" by Eric Ries. URL: "https://www.pdfdrive.com/the-lean-startup-how-todays-entrepreneurs-use-continuous-innovation-to-create-radically-successful-businesses-e164190650.html"

\bibitem{100}
"The Inmates Are Running the Asylum: Why High Tech Products Drive Us Crazy and How to Restore the Sanity" by Alan Cooper. URL: "https://www.pdfdrive.com/the-inmates-are-running-the-asylum-why-high-tech-products-drive-us-crazy-and-how-to-restore-the-sanity-e174831952.html"

\bibitem{101}
"Observing the User Experience: A Practitioner's Guide to User Research" by Mike Kuniavsky. URL: "https://shop.elsevier.com/books/observing-the-user-experience/goodman/978-0-12-384869-7"

\bibitem{102}
"Usability Testing Essentials: Ready, Set...Test!" by Carol M. Barnum. URL: "https://books-library.net/files/books-library.net-08262151Yj5Q9.pdf"

\bibitem{103}
"Designing Interfaces: Patterns for Effective Interaction Design" by Jenifer Tidwell. URL : "https://www.amazon.com/Designing-Interfaces-Patterns-Effective-Interaction/dp/1492051969"

\bibitem{104}
"Mobile Usability" by Jakob Nielsen and Raluca Budiu. URL : "https://moodle2.units.it/pluginfile.php/195427/mod_resource/content/1/Nielsen_Budiu_2012_mobile.pdf"

\bibitem{105}
"Designing Bots: Creating Conversational Experiences" Amir Shevat. URL : "https://www.pdfdrive.com/designing-bots-creating-conversational-experiences-e195069358.html"

\bibitem{106}
Voicebot and Chatbot Design [Электронный ресурс]. URL : "https://subscription.packtpub.com/book/business-and-other/9781789139624/1/ch01lvl1sec12/the-stack-of-conversational-ui"

\bibitem{107}
"Hooked: How to Build Habit-Forming Products" Nir Eyal. URL: "https://www.mann-ivanov-ferber.ru/assets/files/bookparts/pokupatel_na_kryuchke/pokupatel_na_kryuchke_read.pdf"

\bibitem{108}
"Seductive Interaction Design: Creating Playful, Fun, and Effective User Experiences" Stephen Anderson. URL : "https://www.pdfdrive.com/seductive-interaction-design-creating-playful-fun-and-effective-user-experiences-voices-that-matter-e193187842.html"

\bibitem{109}
"Designing Web Interfaces: Principles and Patterns for Rich Interactions" Bill Scott, Theresa Neil. URL : "https://freepdfbook.com/designing-web-interfaces-principles-patterns-rich-interactions/"

\bibitem{110}
Вевюрко Валерия Александровна, Телепченкова Ирина Анатольевна АНАЛИЗ СОЦИАЛЬНОЙ СЕТИ TELEGRAM КАК СРЕДСТВА МАССОВОЙ ИНФОРМАЦИИ И КОММУНИКАЦИИ // Материалы Афанасьевских чтений. 2020. №2 (31). URL: https://cyberleninka.ru/article/n/analiz-sotsialnoy-seti-telegram-kak-sredstva-massovoy-informatsii-i-kommunikatsii (дата обращения: 17.12.2023).


\end{thebibliography}

\end{document}
\documentclass{article}

% Language setting
% Replace `english' with e.g. `spanish' to change the document language
\usepackage[english,russian]{babel}

% Set page size and margins
% Replace `letterpaper' with `a4paper' for UK/EU standard size
\usepackage[letterpaper,top=2cm,bottom=2cm,left=3cm,right=3cm,marginparwidth=1.75cm]{geometry}

% Useful packages
\usepackage{amsmath}
\usepackage{graphicx}
\usepackage[colorlinks=true, allcolors=blue]{hyperref}

\title{Анализ поведения пользователей в текстовых ассистентах и выработка методологий оптимального проектирования пользовательского интерфейса и сценариев}
\author{Калистый Н.С.}

\begin{document}
\maketitle

\begin{abstract}
Работа, которая изменит мир. Мир текстовых ассистентов.
Данная исследовательская работа представляет собой важный вклад в активно развивающееся направление программирования и разработки решений для автоматизации бизнес-процессов с использованием текстовых чат-ботов. В современном мире чат-боты становятся все более популярными благодаря своей универсальности и доступности, не требуя от пользователей установки дополнительных приложений.
Однако, несмотря на актуальность данной темы, практические исследования в этой области остаются недостаточно распространенными. Исследование предлагает оценить эффективность текстовых чат-ботов для автоматизации бизнес-процессов на основе анализа фактических данных, полученных в результате работы трех выпущенных в релиз продуктов. Эмпирический метод исследования, основанный на анализе логов приложения, позволяет оценить реальное воздействие чат-ботов на пользователей.
\end{abstract}

\section{Введение}
Современные технологии информационной обработки и обмена данными значительно изменили повседневную жизнь и бизнес-процессы. В наше время цифровые помощники, такие как текстовые чат-боты, стали неотъемлемой частью взаимодействия между человеком и информацией. Эти боты стали мощным средством автоматизации бизнес-процессов и обеспечения качественного обслуживания клиентов.

В данной работе рассматривается актуальная тема анализа поведения пользователей в текстовых ассистентах и разработки методологий оптимального проектирования интерфейсов и сценариев. Анализ поведения пользователей означает более глубокое понимание того, как пользователи взаимодействуют с ботами, какие запросы они делают, какие ошибки допускают, и в каких местах интерфейса они испытывают затруднения.

Однако, разработка эффективных и удовлетворяющих потребности пользователей чат-ботов включает в себя не только анализ, но и активное участие в процессе проектирования интерфейсов и сценариев. Это означает формирование набора правил и рекомендаций, которые определяют, как бот будет взаимодействовать с пользователями, какие сообщения он будет отправлять, какие кнопки и элементы управления будет использовать, и многое другое.

Данная работа нацелена на изучение этой сложной и многогранной проблематики на примере разработки приложений для автоматизации бизнес-процессов с использованием текстовых ассистентов, доступных через Telegram. В этом контексте, исследование включает в себя как анализ поведения пользователей, так и разработку методологий оптимального проектирования интерфейсов и сценариев. Кроме того, данная работа также обращает внимание на использование нейронных сетей для интеллектуального анализа пользовательского ввода и улучшения качества обслуживания.
\section{Актуальность}
Интернет-сервисы и бизнес-приложения с текстовыми ассистентами стали важной составной частью современного информационного ландшафта. Они используются в таких областях, как клиентский сервис, маркетинг, финансы, здравоохранение и многие другие. Но для того чтобы сделать их более эффективными и удовлетворяющими потребности пользователей, необходимо более глубокое понимание того, как пользователи с ними взаимодействуют и какие проблемы они могут столкнуться.

В настоящее время использование чат-ботов становится особенно актуальным как для обычных пользователей, так и для предприятий и бизнеса. Это обусловлено несколькими факторами, которые делают чат-боты важными и выгодными.

Во-первых, для обычных пользователей использование чат-ботов представляет собой удобное и легкодоступное средство общения с различными сервисами и организациями. Одним из ключевых преимуществ чат-ботов является то, что они не требуют установки дополнительных приложений на устройства пользователя. Это делает процесс взаимодействия максимально удобным и доступным.

Во-вторых, чат-боты обладают широким спектром функциональных возможностей. Они способны автоматизировать разнообразные задачи, начиная от оформления заказов и предоставления консультаций, и заканчивая решением сложных бизнес-процессов. Это позволяет как обычным пользователям, так и бизнес-организациям значительно сэкономить время и ресурсы.

Следует отметить, что для бизнеса, особенно для малых и средних предприятий, чат-боты становятся настоящими инструментами экономии. Разработка и сопровождение текстовых ассистентов оказывается гораздо более дешевой альтернативой по сравнению с созданием и постоянным обновлением мобильных приложений под разные операционные системы, такие как Android и iOS.
Таким образом, чат-боты сегодня являются неотъемлемой частью современной информационной среды, обеспечивая удобство, функциональность и экономичность как для обычных пользователей, так и для предприятий и бизнеса.
\section{Новизна}
Данное исследование отличается от большинства исследований в области разработки чат-ботов тем, что оно не ограничивается конкретной платформой для создания ботов. Вместо этого, оно предоставляет общие рекомендации, которые будут полезны всем разработчикам чат-ботов, независимо от используемых инструментов. Это делает работу более универсальной и применимой для широкого круга специалистов в области разработки.

Важно подчеркнуть, что в данной работе понятие "чат-бот" рассматривается гораздо шире, чем просто "болталка". Чат-боты рассматриваются как полноценные бэкенд-приложения, способные автоматизировать сложные бизнес-процессы. Они служат средством снижения затрат бизнеса на взаимодействие с клиентами, улучшения качества обслуживания и оптимизации рабочих процессов.

В данной работе рассматривается достаточно сложный функционал, который типичные чат-боты обычно не включают в свой набор возможностей. В работе рассматривается интеллектуальный анализ пользовательского ввода с использованием нейросетей, анализ поведения пользователей в приложении и оптимизацию сценариев взаимодействия. Такой подход делает исследование особенно ценным для разработчиков, стремящихся создать более сложные и функциональные чат-боты, способные решать более широкий спектр задач.

Важным аспектом новизны данной работы является акцент на практических результатах. Анализируемые в работе данные представляют собой информацию о реальных пользователях и клиентах бизнеса. Полученные результаты и рекомендации основаны на фактических данных о поведении пользователей, что делает исследование не только актуальным, но и практически применимым для компаний, занимающихся разработкой и внедрением чат-ботов в бизнес-процессы.

В целом, данное исследование представляет собой уникальный вклад в область разработки чат-ботов, который охватывает широкий спектр тем и предоставляет практические рекомендации для улучшения качества и эффективности чат-ботов как инструмента для автоматизации бизнес-процессов.
\section{Обзор литературы}
В работе \cite{greenwade93} сделано все плохо
\section{Описание работы}
1. Сбор и обработка данных:

Для начала проекта необходимо обзавестись исходными данными - логами приложений. Мы используем программу аналитики работы приложений под названием TockBot для сбора данных. Логи приложения представляют собой массив огромных JSON-файлов, каждая строка которых содержит ценную информацию. В каждой строке фиксируется идентификатор бота, в котором произошло событие, идентификатор пользователя, выполнившего действие, ввод, совершенный им, а также информация о состояниях (вершинах графа), между которыми произошел переход.

Лог также включает ответ чат-бота, который может содержать текстовые сообщения, кнопки, причем в анализируемом примере чат ботов в Telegram чат-боты поддерживают два вида кнопок: Inline, которые прикрепляются к сообщению, и Reply, которые отображаются внизу графического интерфейса, под полем текстового ввода и служат пользователю для быстрого ответа боту без необходимости писать ввод вручную.

Лог также включает отправленные ботом изображения, документы и другие реакции. Важным атрибутом каждого лога является timestamp, который фиксирует дату и время события. Для анализа планируется использовать несколько тысяч записей из трех различных проектов, уже запущенных в работу и имеющих данные от реальных пользователей.

2. Свод данных в CSV таблицу:

Полученные логи необходимо структурировать и преобразовать в более удобный формат для анализа. Мы сводим все полученные данные в CSV таблицу, что упрощает работу с ними и позволяет проводить более глубокий анализ.

3. Визуализация графов сценариев:

Для наглядного представления сценариев взаимодействия пользователей с чат-ботами, планируется использовать различные инструменты визуализации данных. Это позволит нам лучше понять структуру чат-бота и выделить основные пути взаимодействия.

4. Анализ данных и их нормализация:

На этом этапе мы проводим анализ данных, включая нормализацию и устранение выбросов. Дополнительно строим карту частот, которая помогает нам выявить популярные пути на графе взаимодействия пользователей. Путем критериальной оценки мы выявляем те пути, которые наиболее востребованы, а также идентифицируем пути, которые можно оптимизировать для улучшения функциональности чат-бота и снижения затрат на его обслуживание.

Корреляционный анализ проводится для определения связей между количеством пользователей, завершивших воронку, и их активностью после этого. Такой анализ может помочь выявить, какие действия пользователей влияют на их долгосрочное использование чат-бота.

5. Исследование с использованием нейронных сетей:

Далее, мы переходим к более глубокому анализу результатов. Используя нейронные сети, мы стремимся выявить общие тенденции в поведении пользователей, их предпочтения и моменты, в которых пользователи отваливаются из воронки. Мы также изучаем сложности, с которыми они сталкиваются, и какие действия могут привести к спаму или нежелательному поведению. Анализируя данные, полученные после анализа нейронными сетями, мы выявляем факторы, которые влияют на успешное завершение воронки.

6. Формулирование гипотез и рекомендаций:

Исходя из результатов исследования, мы формулируем гипотезы и рекомендации. Это включает в себя стратегии по увеличению количества пользователей, завершивших воронку, а также меры для привлечения пользователей, вернувшихся повторно после завершения услуги. Мы также предоставляем практические советы о том, какие методы разработки могут сократить затраты на создание чат-ботов и повысить их понятность для пользователей.

7. Обратная связь и реализация рекомендаций:

Завершая этот проект, мы обмениваемся результатами с заказчиком и получаем обратную связь о результатах, полученных после внедрения предложенных рекомендаций и методологий в исследованные проекты. Это важный этап, который позволяет убедиться в эффективности предложенных мер и внести коррективы в дальнейшей разработке и оптимизации чат-ботов.
\section{Практическая ценность}
Исследование, проведенное в ходе данной работы, имеет высокую практическую ценность как для разработчиков, так и для заказчиков чат-ботов. В процессе исследования были получены ценные инсайты и рекомендации, которые способствуют оптимизации и улучшению сценариев взаимодействия пользователей с чат-ботами.

Одним из важных достижений является установление наиболее востребованных направлений движения пользователей по графу сценариев чат-бота. Это позволило выявить ключевые точки взаимодействия и сделать акцент на них, улучшая пользовательский опыт.

В рамках исследования были разработаны рекомендации по оптимизации сценариев, направленные на сокращение количества вершин графа без потери функциональности приложения чат-бота. Такие оптимизации позволяют сделать интерфейс более интуитивным и удобным для пользователей, снижая вероятность недопонимания и ошибок.

Другим важным результатом исследования стало предложение упрощенных и более ясных формулировок, избегание сложных аббревиатур и сокращений. Эти рекомендации способствуют легкости восприятия информации и обеспечивают четкое взаимодействие пользователей с чат-ботом.

Одним из наиболее заметных достижений данного исследования является разработка методологий, которые позволяют компаниям-заказчикам создавать текстовых ассистентов с учетом всех выявленных рекомендаций. Это позволяет сразу создавать чат-ботов без указанных ошибок, с оптимизированными сценариями и интерфейсом, что сокращает стоимость разработки каждого последующего чат-бота примерно на 15-30%, в зависимости от сложности проекта и числа функциональных возможностей.

Компания-заказчик исследования пришла к положительным выводам от результатов проведенной работы. Полученные рекомендации и методологии уже на практике помогли сократить затраты и сделать взаимодействие с чат-ботами более продуктивным и приятным для пользователей.
\section{Заключение}
Результаты данной работы будут иметь высокую практическую ценность для разработчиков и дизайнеров чат-ботов, а также для бизнес-аналитиков и менеджеров по обслуживанию клиентов. Исследование позволит оптимизировать процессы разработки и существенно улучшить качество обслуживания пользователей через текстовые ассистенты. Кроме того, разработанные методологии могут быть использованы как основа для дальнейших исследований и практических применений в области чат-ботов и искусственного интеллекта.

В ходе проведенного исследования было выявлено, что разработка и оптимизация чат-ботов представляют собой важное направление для современных бизнес-процессов. Чат-боты являются мощным инструментом автоматизации и оптимизации бизнес-взаимодействия с пользователями, и их актуальность постоянно растет.

Исследование показало, что для создания успешных чат-ботов, сочетающих в себе функциональность, легкость в использовании и высокий уровень понимания пользователей, необходимо комплексное исследование и разработка. Это включает в себя работу не только программистов, но и специалистов по дизайну диалоговых ассистентов, UX/UI дизайнеров, бизнес-аналитиков и других экспертов.

Для достижения оптимальных результатов в создании чат-ботов необходимо уделить внимание не только технической реализации, но и пониманию потребностей и ожиданий пользователей. Этот процесс требует интегрированного подхода, где каждый участник команды вносит свой вклад в создание более эффективных и удовлетворяющих решений.

Исследование показало, что оптимальный баланс между функциональностью и понятностью пользователю может быть достигнут только при сотрудничестве специалистов из разных областей. Целая команда, включающая в себя программистов, дизайнеров диалоговых ассистентов, UX/UI дизайнеров, бизнес-аналитиков и других экспертов, способна создавать чат-боты, которые не только автоматизируют бизнес-процессы, но и обеспечивают легкость и комфорт взаимодействия для пользователей.

В итоге, данное исследование подтвердило значимость чат-ботов в современном мире бизнеса и подчеркнуло важность интеграции различных специалистов для создания ботов, которые будут наиболее эффективными и удовлетворяющими запросы пользователей.

\bibliographystyle{alpha}
\bibliography{sample}

\end{document}
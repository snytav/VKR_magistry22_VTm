\documentclass{article}

% Language setting
% Replace `english' with e.g. `spanish' to change the document language
\usepackage[english,russian]{babel}

% Set page size and margins
% Replace `letterpaper' with `a4paper' for UK/EU standard size
\usepackage[letterpaper,top=2cm,bottom=2cm,left=3cm,right=3cm,marginparwidth=1.75cm]{geometry}

% Useful packages
\usepackage{amsmath}
\usepackage{graphicx}
\usepackage[colorlinks=true, allcolors=blue]{hyperref}

\title{Название ВКР}
\author{Назаров И.С.}

\begin{document}
\maketitle

\begin{abstract}
Your abstract.
\end{abstract}


Цель данной практики состоит в том, чтобы на примере конкретного 
исследования показать, как можно использовать математические методы 
дополнения выборок малых групп до размеров, позволяющих применять 
методы классического статистического анализа. Для этого были 
проведены опросы 20 преподавателей университета с целью выявления 
потребности во внедрении новых технологий. Но объем выборки составил всего 20 результатов, что не позволяет получить достоверные результаты 
при применении методов классической статистики.


\section{Актуальность}
Анализ поведения пользователей в текстовых ассистентах и выработка методологий оптимального проектирования пользовательского интерфейса и сценариев" является актуальной по нескольким причинам.

Во-первых, с развитием технологий и искусственного интеллекта текстовые ассистенты, такие как Siri, Alexa, Google Assistant и другие, стали все более распространенными и широко используемыми. Они помогают людям выполнять различные задачи, создавая удобное и интуитивное взаимодействие с компьютером или другими устройствами. Понимание поведения пользователей в таких системах и разработка оптимальных методологий их проектирования являются важными аспектами для улучшения качества пользовательского опыта.

Во-вторых, анализ поведения пользователей в текстовых ассистентах позволяет исследовать, как люди взаимодействуют с такими системами, какие команды они используют, какие проблемы возникают во время использования и т. д. Такой анализ может помочь выявить причины ошибок или неудобств, которые пользователи испытывают при работе с текстовыми ассистентами, а также предложить улучшения и новые функциональности для повышения их эффективности и удовлетворенности пользователей.

В-третьих, оптимальное проектирование пользовательского интерфейса и сценариев играет ключевую роль в создании удобной и понятной системы. Разработка методологий и стандартов для проектирования интерфейсов текстовых ассистентов может помочь улучшить взаимодействие между пользователем и системой, сделать его более интуитивным и эффективным. Это особенно важно в случае текстовых ассистентов, где пользователи взаимодействуют с системой через набор текстовых команд или запросов.

Таким образом, актуальность ВКР на данную тему обусловлена широким распространением текстовых ассистентов и необходимостью улучшения их пользовательского опыта. Анализ поведения пользователей и разработка оптимальных методологий проектирования пользовательского интерфейса и сценариев позволят создать более удобные и эффективные системы, которые лучше будут соответствовать потребностям пользователей и повысить их удовлетворенность и эффективность при работе с текстовыми ассистентами.
\section{Новизна}

Проведение глубокого анализа поведения пользователей в текстовых ассистентах. Данный анализ включает исследование привычек, предпочтений и ожиданий пользователей, а также оценку эффективности и удовлетворенности ими от использования таких ассистентов. Проведение такого обширного и детального исследования является новизной, поскольку подобные исследования до сих пор не были широко осуществлены.

Выработка методологий оптимального проектирования пользовательского интерфейса текстовых ассистентов. Это включает в себя разработку новых подходов к улучшению удобства использования, организации информации и взаимодействия с пользователем. Разработка таких методологий для текстовых ассистентов является относительно новой областью и представляет собой важную и актуальную задачу в современной науке.

Создание оптимальных исследовательских сценариев. В процессе исследования поведения пользователей в текстовых ассистентах требуется разработка сценариев взаимодействия между пользователем и ассистентом. Основываясь на полученных данных и наблюдениях о поведении пользователей, новизна заключается в создании оптимальных исследовательских сценариев, которые максимально точно отражают реальные ситуации и потребности пользователей.
\section{Обзор литературы}
В работе \cite{greenwade93} сделано все плохо
\section{Описание работы}
\section{Практическая ценность}
\section{Заключение}


\bibliographystyle{alpha}
\bibliography{sample}
   
Ураев Денис Алексеевич Метрики для оценки качества чат-бот приложений // Наука, техника и образование. 2019. №9 (62). URL: https://cyberleninka.ru/article/n/metriki-dlya-otsenki-kachestva-chat-bot-prilozheniy (дата обращения: 17.12.2023).

\bibitem{1}
Махина Екатерина Дмитриевна, Пальчунов Дмитрий Евгеньевич Программная система для определения речевых действий в текстах естественного языка // Вестник НГУ. Серия: Информационные технологии. 2018. №4. URL: https://cyberleninka.ru/article/n/programmnaya-sistema-dlya-opredeleniya-rechevyh-deystviy-v-tekstah-estestvennogo-yazyka (дата обращения: 17.12.2023).

\bibitem{2}
Гольчевский Ю.В., Непеин А.В. ПРОЕКТИРОВАНИЕ И РАЗРАБОТКА ЧАТ-БОТА ДЛЯ ПРЕДСТАВЛЕНИЯ РАСПИСАНИЯ В СОЦИАЛЬНОЙ СЕТИ // Вестник Сыктывкарского университета. Серия 1. Математика. Механика. Информатика. 2021. №3 (40). URL: https://cyberleninka.ru/article/n/proektirovanie-i-razrabotka-chat-bota-dlya-predstavleniya-raspisaniya-v-sotsialnoy-seti (дата обращения: 17.12.2023).

\bibitem{3}
"Designing with the Mind in Mind: Simple Guide to Understanding User Interface Design Rules" by Jeff Johnson. URL: "https://lib.fbtuit.uz/assets/files/Designing-withtheMindinMindSimple-Johnson-Kaufmann2010.pdf"

\bibitem{4}
"Don't Make Me Think, Revisited: A Common Sense Approach to Web Usability" by Steve Krug. URL:
"https://pdflake.com/wp-content/uploads/2022/09/Dont-Make-Me-Think-PDF.pdf"

\bibitem{5}
"The Design of Everyday Things" by Don Norman. URL:
"https://streettrotter.com/wp-content/uploads/2020/08/The-Design-of-Everyday-Things-Book-by-Don-Norman.pdf"

\bibitem{6}
"Measuring the User Experience: Collecting, Analyzing, and Presenting Usability Metrics" by William Albert. URL: "https://medium.com/@GemmaAntho98100/epub-measuring-the-user-experience-collecting-analyzing-and-presenting-usability-metrics-f923d548dc"

\bibitem{7}
"Seductive Interaction Design: Creating Playful, Fun, and Effective User Experiences" by Stephen Anderson. URL: "https://www.pdfdrive.com/seductive-interaction-design-creating-playful-fun-and-effective-user-experiences-voices-that-matter-e193187842.html"

\bibitem{8}
"Hooked: How to Build Habit-Forming Products" by Nir Eyal. URL: "https://medium.com/@nansyonlinebusinnes/book-review-hooked-how-to-build-habit-forming-products-by-nir-eyal-1a2c4deeb43a"

\bibitem{9}
"Usability Engineering" by Jakob Nielsen. URL: "https://drive.google.com/file/d/0B5tR1YhNBlD2Wlg5NW1fS09sOFU/view?usp=sharing"

\bibitem{10}
"Lean UX: Designing Great Products with Agile Teams" by Jeff Gothelf. URL:
"https://www.pdfdrive.com/lean-ux-designing-great-products-with-agile-teams-e185409456.html"

\bibitem{11}
"The Elements of User Experience: User-Centered Design for the Web and Beyond" by Jesse James Garrett. URL: "https://www.pdfdrive.com/the-elements-of-user-experience-user-centered-design-for-the-web-and-beyond-2nd-edition-voices-that-matter-e157223396.html"

\bibitem{12}
"100 Things Every Designer Needs to Know About People" by Susan Weinschenk. URL: "https://www.pdfdrive.com/100-things-every-designer-needs-to-know-about-people-what-makes-them-tick-e156711199.html"

\bibitem{13}
"User Story Mapping: Discover the Whole Story, Build the Right Product" by Jeff Patton. URL: https://www.pdfdrive.com/user-story-mapping-discover-the-whole-story-build-the-right-product-e157789213.html

\bibitem{14}
"Designing for Interaction: Creating Innovative Applications and Devices" by Dan Saffer. URL: "https://books.google.ru/books?id=Dd3Hcs9jeoUC&hl=ru"

\bibitem{15}
"About Face: The Essentials of Interaction Design" by Alan Cooper. URL: "https://medium.com/@6cuongseven9899g/about-face-the-essentials-of-interaction-design-full-acces-c0ddd099e9a1"

\bibitem{16}
"Information Architecture: For the Web and Beyond" by Louis Rosenfeld and Peter Morville. URL: "https://www.pdfdrive.com/information-architecture-for-the-web-and-beyond-e158738770.html"

\bibitem{17}
"The Lean Startup: How Today's Entrepreneurs Use Continuous Innovation to Create Radically Successful Businesses" by Eric Ries. URL:
"https://www.pdfdrive.com/the-lean-startup-how-todays-entrepreneurs-use-continuous-innovation-to-create-radically-successful-businesses-e164190650.html"

\bibitem{18}
"The Inmates Are Running the Asylum: Why High Tech Products Drive Us Crazy and How to Restore the Sanity" by Alan Cooper. URL: "https://www.pdfdrive.com/the-inmates-are-running-the-asylum-why-high-tech-products-drive-us-crazy-and-how-to-restore-the-sanity-e174831952.html"

\bibitem{19}
"Observing the User Experience: A Practitioner's Guide to User Research" by Mike Kuniavsky. URL: "https://shop.elsevier.com/books/observing-the-user-experience/goodman/978-0-12-384869-7"

\bibitem{20}
"Usability Testing Essentials: Ready, Set...Test!" by Carol M. Barnum. URL: "https://books-library.net/files/books-library.net-08262151Yj5Q9.pdf"

\bibitem{21}

С.А. Белоусова, Ю.И. Рогозов
АНАЛИЗ ПОДХОДОВ К СОЗДАНИЮ ПОЛЬЗОВАТЕЛЬСКОГО
ИНТЕРФЕЙСА* . https://cyberleninka.ru/article/n/analiz-podhodov-k-sozdaniyu-polzovatelskogo-interfeysa
УПРАВЛЕНИЕ ПРОЕКТИРОВАНИЕМ И РЕАЛИЗАЦИЕЙ ПОЛЬЗОВАТЕЛЬСКПГН ИНТЕРФЕЙСА НА ОСНОВЕ ОНТОЛОГИЙ1
\bibitem{22}

В. В. Грибова, А. С. Клещев. https://cyberleninka.ru/article/n/upravlenie-proektirovaniem-i-realizatsiey-polzovatelskogo-interfeysa-na-osnove-ontologiy

\bibitem{23}
 Н.В. НОВОЖИЛОВА
ОСОБЕННОСТИ ПРОЕКТИРОВАНИЯ ДРУЖЕСТВЕННЫХ ИНТЕРФЕЙСОВ ДЛЯ ПОЛЬЗОВАТЕЛЕЙ-ЭКОНОМИСТОВ*
https://cyberleninka.ru/article/n/osobennosti-proektirovaniya-druzhestvennyh-interfeysov-dlya-polzovateley-ekonomistov

\bibitem{24}
КАТЕГОРИЯ «ЕСТЕСТВЕННОСТЬ» В КЛАССИФИКАЦИИ ПОЛЬЗОВАТЕЛЬСКИХ ИНТЕРФЕЙСОВ
Н.Н. Зильберман, С.А. Алексеев
https://cyberleninka.ru/article/n/kategoriya-estestvennost-v-klassifikatsii-polzovatelskih-interfeysov

\bibitem{25}
УНИФИЦИРОВАННАЯ МОДЕЛЬ ПОЛЬЗОВАТЕЛЬСКИХ ИНТЕРФЕЙСОВ
ИНТЕЛЛЕКТУАЛЬНЫХ СИСТЕМ
Д.Г. КОЛБ
https://cyberleninka.ru/article/n/unifitsirovannaya-model-polzovatelskih-interfeysov-intellektualnyh-sistem

\bibitem{26}
ИСПОЛЬЗОВАНИЕ СБОРОЧНОЙ ТЕХНОЛОГИИ ДЛЯ ПОСТРОЕНИЯ ПОЛЬЗОВАТЕЛЬСКИХ ИНТЕРФЕЙСОВ СЕТЕВОЙ ИНФОРМАЦИОННО-ВЫЧИСЛИТЕЛЬНОЙ
СИСТЕМЫ
С. В. Пискунов, С. В. Кратов, М. Б. Остапкевич, А. В. Веселов*
https://cyberleninka.ru/article/n/ispolzovanie-sborochnoy-tehnologii-dlya-postroeniya-polzovatelskih-interfeysov-setevoy-informatsionno-vychislitelnoy-sistemy

\bibitem{27}
 ЕСТЕСТВЕННО-ЯЗЫКОВЫЕ ИНТЕРФЕЙСЫ ИНТЕЛЛЕКТУАЛЬНЫХ ВОПРОСНО-ОТВЕТНЫХ СИСТЕМ
В.А. ЖИТКО1, В Н. ВЯЛЬЦЕВ2, Ю.С. ГЕЦЕВИЧ2, А.А. КУЗЬМИН3
https://cyberleninka.ru/article/n/estestvenno-yazykovye-interfeysy-intellektualnyh-voprosno-otvetnyh-sistem

\bibitem{28}
ЕСТЕСТВЕННО-ЯЗЫКОВОЙ ПОЛЬЗОВАТЕЛЬСКИЙ ИНТЕРФЕЙС
ДИАЛОГОВОЙ СИСТЕМЫ
Р.В. Посевкин, И.А. Бессмертный https://cyberleninka.ru/article/n/estestvenno-yazykovoy-polzovatelskiy-interfeys-dialogovoy-sistemy

\bibitem{29}
 Л.Е. Малыгина
ЧАТ-БОТЫ И ИСКУССТВЕННЫЙ ИНТЕЛЛЕКТ: ПЕРСПЕКТИВЫ РАЗВИТИЯ ТЕЛЕВИЗИОННОГО ПРОМОДИСКУРСА 
https://cyberleninka.ru/article/n/chat-boty-i-iskusstvennyy-intellekt-perspektivy-razvitiya-televizionnogo-promodiskursa

\bibitem{30}
The Role of BAs in User Interface Design [Electronic re  sourse] //Business analyst learnings [site]. URL: https://busi  nessanalystlearnings.com/blog/2014/7/29/the-role-of-basin-user-interface-design (accessed: 22.04.2017). 

\bibitem{31}
Леоненков А. Самоучитель UML 2. СПб. : БХВ Петербург, 2007. 558 с.
 Вендров А.М. Проектирование программного обеспечения экономических информационных систем. М. : Финансы и статистика, 2005. 544 с.

 \bibitem{32}
Aлан Купер об интерфейсе. Основы проектирования взаи  модействия / пер. с англ. СПб. : Символ-Плюс, 2009. 688 с.

\bibitem{33}
Андреев В. О чем надо помнить при разработке пользователь  ского интерфейса [Электронный ресурс] //Usability в России [сайт]. URL: http://www.usability.ru/Articles/instruction. htm (дата обращения: 22.04.2017).

\bibitem{34}
 Раскин Дж. Интерфейс: новые направления в проектирова  нии компьютерных систем. М. : Символ-Плюс, 2005. 160 с.
 Андрейчиков А. В., Андрейчикова О. Н. Интеллектуаль  ные информационные системы. М. : Финансы и статисти  ка, 2004. 422 с.

 \bibitem{35}
Головач В. Дизайн пользовательского интерфейса. Ис  кусство мыть слона [Электронный ресурс] //Юзетикс [сайт]. URL: http://www.usethics.ru/lib (дата обраще  ния: 28.04.2017).
Торес Р. Дж. Практическое руководство по проектирова  нию и разработке пользовательского интерфейса / пер. с англ. М. : Вильямс, 2002. 400 с.

\bibitem{36}
 Жарков С. Shareware: профессиональная разработка и про  движение программ. СПб. : БХВ Петербург, 2002. 320 c.

 \bibitem{37}
 Руководство по WPF [Электронный ресурс] //Сайт о программировании [сайт]. URL: https://metanit. com/sharp/wpf (дата обращения: 28.04.2017). 
21Протокол заседания Совета по стратегическому развитию и приоритетным проектам «О программе «Цифровая экономика» от 5 июля 2017 г. (2017) 

\bibitem{38}
 Цифровая Россия: новая реальность (2017) / McKinsey. http://www.tadviser.ru/images/c/c2/ Digital-Russia-report.pdf.

\bibitem{39}
 В чем разница между искусственным интеллектом, машинным обучением и глубоким обучением? (2016) / Nvidia. http://www.nvidia.ru/ object/whats-difference-ai-machine-learning-deeplearning-blog-ru.html. 

\bibitem{40}
World Economic Forum Annual Meeting 2016: Mastering the Fourth Industrial Revolution (2016) / World Economic Forum. 02.02.2016. https://www.weforum.org/reports/world-economicforum-annual-meeting 2016-mastering-the-fourthindustrial-revolution. 

\bibitem{41}
Люггер Дж.Ф. (2004) Искусственный интеллект: стратегии и методы решения сложных проблем. М.: Издательский дом «Вильямс». 864 с.

\bibitem{42}
Искусственный интеллект (ИИ) / Artificial Intelligence (AI) как ключевой фактор цифровизации глобальной экономики (2017) / CRN/RE. 24.02.2017. https://www.crn.ru/news/detail.php? ID=117544.

\bibitem{43}
Екатерина Петровна Турбина Дистанционные технологии в преподавании иностранных языков (на примере приложения telegram) // Вестник Шадринского государственного педагогического университета. 2023. №1 (57). URL: https://cyberleninka.ru/article/n/distantsionnye-tehnologii-v-prepodavanii-inostrannyh-yazykov-na-primere-prilozheniya-telegram (дата обращения: 17.12.2023).

\bibitem{44}
Валинурова Анна Александровна, Балабанова Наталья Владимировна, Маценков Иван Алексеевич АЛГОРИТМ РАЗРАБОТКИ TELEGRAM -БОТА - ПРОДУКТИВНОГО ПОМОЩНИКА СОВРЕМЕННОГО БИЗНЕСА // Современные наукоемкие технологии. Региональное приложение . 2023. №2 (74). URL: https://cyberleninka.ru/article/n/algoritm-razrabotki-telegram-bota-produktivnogo-pomoschnika-sovremennogo-biznesa (дата обращения: 17.12.2023).

\bibitem{45}
Шумилина Мария Александровна, Коробко Анна Владимировна РАЗРАБОТКА ЧАТ-БОТА НА ЯЗЫКЕ ПРОГРАММИРОВАНИЯ PYTHON В МЕССЕНДЖЕРЕ "TELEGRAM" // Научные известия. 2022. №28. URL: https://cyberleninka.ru/article/n/razrabotka-chat-bota-na-yazyke-programmirovaniya-python-v-messendzhere-telegram (дата обращения: 17.12.2023).

\bibitem{46}
Белых Ольга Александровна ОБРАЗОВАТЕЛЬНЫЙ КОНТЕНТ МЕССЕНДЖЕРА «TELEGRAM» КАК ИНСТРУМЕНТ ПОВЫШЕНИЯ РЕЗУЛЬТАТИВНОСТИ ОБРАЗОВАТЕЛЬНОГО ПРОЦЕССА // Педагогический ИМИДЖ. 2021. №3 (52). URL: https://cyberleninka.ru/article/n/obrazovatelnyy-kontent-messendzhera-telegram-kak-instrument-povysheniya-rezultativnosti-obrazovatelnogo-protsessa (дата обращения: 17.12.2023).

\bibitem{47}
Верещагина Е.А., Рудниченко А.К., Рудниченко Д.С. МОНИТОРИНГ СТАБИЛЬНОСТИ СЕТЕВОЙ ИНФРАСТРУКТУРЫ С ИСПОЛЬЗОВАНИЕМ МЕССЕНДЖЕРА TELEGRAM // ИВД. 2020. №11 (71). URL: https://cyberleninka.ru/article/n/monitoring-stabilnosti-setevoy-infrastruktury-s-ispolzovaniem-messendzhera-telegram (дата обращения: 17.12.2023).

\bibitem{48}
Сунцова Д. И., Павлов В. А., Макаренко З. В., Бахолдин П. П., Полицинский А. С., Кремлев А. С., Маргун А. А. РАЗРАБОТКА TELEGRAM-БОТА «ОПРЕДЕЛЕНИЕ УРОВНЯ ГОТОВНОСТИ ТЕХНОЛОГИИ» // Экономика науки. 2022. №1. URL: https://cyberleninka.ru/article/n/razrabotka-telegram-bota-opredelenie-urovnya-gotovnosti-tehnologii (дата обращения: 17.12.2023).

\bibitem{49}
Ro’Ziqulova, Maqsuda Abriy Qizi INTELLEKTUAL TEST SINOVLARINI O‘TKAZUVCHI TELEGRAM BOT YARATISH // ORIENSS. 2022. №6. URL: https://cyberleninka.ru/article/n/intellektual-test-sinovlarini-o-tkazuvchi-telegram-bot-yaratish (дата обращения: 17.12.2023).

\bibitem{50}
Филатова Зульфия Мирсайжановна, Закирова Нурия Ришатовна СОЗДАНИЕ TELEGRAM-БОТА ДЛЯ АВТОМАТИЗАЦИИ АДМИНИСТРАТИВНОЙ ДЕЯТЕЛЬНОСТИ // Проблемы современного педагогического образования. 2023. №79-4. URL: https://cyberleninka.ru/article/n/sozdanie-telegram-bota-dlya-avtomatizatsii-administrativnoy-deyatelnosti (дата обращения: 17.12.2023).

\bibitem{51}
Краснова Марина Николаевна ИСПОЛЬЗОВАНИЕ КАНАЛА TELEGRAM ПРИ ПОДГОТОВКЕ УЧАЩИХСЯ К ЭКЗАМЕНАМ // Вестник науки и образования. 2022. №7-1 (127). URL: https://cyberleninka.ru/article/n/ispolzovanie-kanala-telegram-pri-podgotovke-uchaschihsya-k-ekzamenam (дата обращения: 17.12.2023).

\bibitem{52}
Коноплев Дмитрий Эдуардович Telegram как новая среда коммуникации в СМИ и соцсетях // Знак: проблемное поле медиаобразования. 2017. №3 (25). URL: https://cyberleninka.ru/article/n/telegram-kak-novaya-sreda-kommunikatsii-v-smi-i-sotssetyah (дата обращения: 17.12.2023).

\bibitem{53}
Лытнева Анна Андреевна, Дубинина Анна Эмировна TELEGRAM КАК НОВАЯ ПЛОЩАДКА В СРЕДСТВАХ МАССОВОЙ КОММУНИКАЦИИ // Материалы Афанасьевских чтений. 2020. №2 (31). URL: https://cyberleninka.ru/article/n/telegram-kak-novaya-ploschadka-v-sredstvah-massovoy-kommunikatsii (дата обращения: 17.12.2023).

\bibitem{54}
Баранова Екатерина Андреевна, Андрианова Дарья Дмитриевна СОВРЕМЕННЫЙ РОССИЙСКИЙ РЫНОК ДЕЛОВОЙ ИНФОРМАЦИИ В УСЛОВИЯХ КОНКУРЕНЦИИ ЕЖЕДНЕВНЫХ СМИ И TELEGRAM-КАНАЛОВ // Litera. 2022. №8. URL: https://cyberleninka.ru/article/n/sovremennyy-rossiyskiy-rynok-delovoy-informatsii-v-usloviyah-konkurentsii-ezhednevnyh-smi-i-telegram-kanalov (дата обращения: 17.12.2023).

\bibitem{55}
Дементьева К. В. РАЗВИТИЕ TELEGRAM-КАНАЛОВ В МЕДИАПРОСТРАНСТВЕ РОССИЙСКИХ РЕГИОНОВ: СПЕЦИФИКА, ТИПОЛОГИЯ, ПЕРСПЕКТИВЫ РАЗВИТИЯ (НА ПРИМЕРЕ TELEGRAM-КАНАЛОВ РЕСПУБЛИКИ МОРДОВИЯ) // Вестник НГУ. Серия: История, филология. 2021. №6. URL: https://cyberleninka.ru/article/n/razvitie-telegram-kanalov-v-mediaprostranstve-rossiyskih-regionov-spetsifika-tipologiya-perspektivy-razvitiya-na-primere-telegram (дата обращения: 17.12.2023).

\bibitem{56}
Абрамова А.И. ИСПОЛЬЗОВАНИЕ TELEGRAM-БОТА В ОБРАЗОВАТЕЛЬНОМ ПРОЦЕССЕ ВУЗА // Вестник науки. 2022. №1 (46). URL: https://cyberleninka.ru/article/n/ispolzovanie-telegram-bota-v-obrazovatelnom-protsesse-vuza (дата обращения: 17.12.2023).

\bibitem{57}
Андриянова Софья Сархановна, Веретено Александра Александровна Использование мессенджера Telegram для продвижения бренда // Economics. 2018. №3 (35). URL: https://cyberleninka.ru/article/n/ispolzovanie-messendzhera-telegram-dlya-prodvizheniya-brenda (дата обращения: 17.12.2023).

\bibitem{58}
Алексеев В.А., Домашнев П.А., Лаврухина Т.В., Назаркин О.А. Использование функций мессенджера Telegram для обмена сообщениями между узлами распределенной вычислительной системы // International Journal of Open Information Technologies. 2019. №6. URL: https://cyberleninka.ru/article/n/ispolzovanie-funktsiy-messendzhera-telegram-dlya-obmena-soobscheniyami-mezhdu-uzlami-raspredelennoy-vychislitelnoy-sistemy (дата обращения: 17.12.2023).

\bibitem{59}
Шакиров Р.И., Татаурова А.С. Автоматизация учебного расписания через Telegram-bot // Символ науки. 2020. №1. URL: https://cyberleninka.ru/article/n/avtomatizatsiya-uchebnogo-raspisaniya-cherez-telegram-bot (дата обращения: 17.12.2023).

\bibitem{60}
Байкова К.Д., Медведева T.A. АНАЛИЗ И РАЗРАБОТКА ФУНКЦИОНАЛЬНОГО TELEGRAM-БОТА // Молодой исследователь Дона. 2021. №6 (33). URL: https://cyberleninka.ru/article/n/analiz-i-razrabotka-funktsionalnogo-telegram-bota (дата обращения: 17.12.2023).

\bibitem{61}
Бийбосунов Б. И., Бийбосунова С. К., Жолочубеков Н. Ж. Описание концепции Telegram ботов и их разработка // Colloquium-journal. 2020. №7 (59). URL: https://cyberleninka.ru/article/n/opisanie-kontseptsii-telegram-botov-i-ih-razrabotka (дата обращения: 17.12.2023).

\bibitem{62}
Назаров Алишер Искендерович, Мухаммадиев Нурфайз РАЗРАБОТКА МОДИФИЦИРОВАННОГО TELEGRAM-БОТА // SAI. 2023. №Special Issue 3. URL: https://cyberleninka.ru/article/n/razrabotka-modifitsirovannogo-telegram-bota (дата обращения: 17.12.2023).

\bibitem{63}
Артамонова Мария Валериевна, Мамбетов Акбар Азаматович, Тулина Екатерина Валерьевна Чат-бот как инструмент в работе переводчика // Litera. 2023. №8. URL: https://cyberleninka.ru/article/n/chat-bot-kak-instrument-v-rabote-perevodchika (дата обращения: 17.12.2023).

\bibitem{64}
Т.А. Дейнеко, Д.А. Бобров ЧАТ-БОТ ВКОНТАКТЕ «РАСПИСАНИЕ ЗАНЯТИЙ ОМГУ» // МСиМ. 2020. №3 (55). URL: https://cyberleninka.ru/article/n/chat-bot-vkontakte-raspisanie-zanyatiy-omgu (дата обращения: 17.12.2023).

\bibitem{65}
Параскевов Александр Владимирович, Каденцева Анастасия Александровна, Мороз Сергей Игоревич Перспективы и особенности разработки чат-ботов // Научный журнал КубГАУ. 2017. №130. URL: https://cyberleninka.ru/article/n/perspektivy-i-osobennosti-razrabotki-chat-botov (дата обращения: 17.12.2023).

\bibitem{66}
Бахтин И.В. РАЗРАБОТКА ЧАТ-БОТОВ ДЛЯ АВТОМАТИЗАЦИИ БИЗНЕС- ПРОЦЕССОВ // Форум молодых ученых. 2020. №2 (42). URL: https://cyberleninka.ru/article/n/razrabotka-chat-botov-dlya-avtomatizatsii-biznes-protsessov (дата обращения: 17.12.2023).

\bibitem{67}
Дулёв А.А. ИННОВАЦИОННЫЕ БАНКОВСКИЕ ПРОДУКТЫ И ИХ РАЗВИТИЕ В РОССИЙСКОЙ ФЕДЕРАЦИИ // Хроноэкономика. 2020. №5 (26). URL: https://cyberleninka.ru/article/n/innovatsionnye-bankovskie-produkty-i-ih-razvitie-v-rossiyskoy-federatsii (дата обращения: 17.12.2023).

\bibitem{68}
Строев Владимир Витальевич, Тихонов Алексей Иванович АНАЛИЗ ИСПОЛЬЗОВАНИЯ ЧАТ-БОТА КАК ИНСТРУМЕНТА ОНЛАЙН-ОБУЧЕНИЯ ПЕРСОНАЛА // Московский экономический журнал. 2022. №6. URL: https://cyberleninka.ru/article/n/analiz-ispolzovaniya-chat-bota-kak-instrumenta-onlayn-obucheniya-personala (дата обращения: 17.12.2023).

\bibitem{69}
Яндыбаева Н.В., Акельев И.В. РАЗРАБОТКА ЧАТ-БОТА ДЛЯ ТЕХНИЧЕСКОЙ ПОДДЕРЖКИ ПОЛЬЗОВАТЕЛЕЙ В МЕДИЦИНСКОМ УЧРЕЖДЕНИИ // Международный журнал гуманитарных и естественных наук. 2022. №9-1. URL: https://cyberleninka.ru/article/n/razrabotka-chat-bota-dlya-tehnicheskoy-podderzhki-polzovateley-v-meditsinskom-uchrezhdenii (дата обращения: 17.12.2023).

\bibitem{70}
Минина Вера Николаевна HR-боты в управлении человеческими ресурсами организации // Вестник Санкт-Петербургского университета. Менеджмент. 2019. №3. URL: https://cyberleninka.ru/article/n/hr-boty-v-upravlenii-chelovecheskimi-resursami-organizatsii (дата обращения: 17.12.2023).






































\end{document}
\documentclass{article}
\usepackage[english,russian]{babel}
\usepackage[letterpaper,top=2cm,bottom=2cm,left=3cm,right=3cm,marginparwidth=1.75cm]{geometry}

% Useful packages
\usepackage{amsmath}
\usepackage{graphicx}
\usepackage[colorlinks=true, allcolors=blue]{hyperref}

\title{РЕГРЕССИОННЫЙ АНАЛИЗ ПРОИЗВОДИТЕЛЬНОСТИ СУПЕР-ЭВМ}
\author{Выполнил ученик гр.22-ВТм Клиновицкий Павел}

\begin{document}
\maketitle

\begin{abstract}
Данная работа посвящена вычислению производительности Суперкомпьютера, он же Супер-ЭВМ, с использованием задач метода частиц в ячейках и последующим расчетам по методам факторного анализа и множественной линейной регрессии при помощи, написанного на языке программирования Python, скрипта, для обработки полученных результатов.

Ключевые слова: Супер-ЭВМ, факторный анализ, множественная линейная регрессия.
\end{abstract}

\section{Введение}
Суперкомпьютер, он же Супер-ЭВМ – это огромная машина, выполняющая триллионы вычислительных операций. В то время, как обычный компьютер имеет один центральный процессор, число ядер которого не превышает и тридцати двух, среднестатистический Суперкомпьютер располагает тысячами, а то и миллионами ядер.

Супер-ЭВМ используется для осуществления трудоемких вычислительных процессов и обработки колоссального объема информации в масштабе реального времени, а также для применения точных моделей исследуемых процессов.

Необходимость анализа производительности Суперкомпьютера возникает на основании заявленных со стороны производителя мощностей, обозначенных в рейтингах top50 и top500, которые, как правило, не соответствуют действительности. Полученные в ходе работы и нагрузочных тестирований статистические расчеты предоставят поле для анализа правдоподобности заявлений.

\section{Актуальность}

\section{Новизна}
Новизна и оригинальность идей, положенных в работу:

Запуск плазменного приложения метода частиц в ячейках, с целью сбора статистической информации о Супер-ЭВМ.

\section{Обзор литературы}
В работе \cite{greenwade93}

\section{Описание работы}
При помощи языка программирования Python написан скрипт для анализов результатов запуска плазменного приложения на Суперкомпьютере;

Вычислена производительности Суперкомпьютера при помощи алгоритмов расчетов: факторного анализа, множественной линейной регрессии.

\section{Научная, научно-техническая и практическая ценность}
Практическая:

повышение качества подбора комплектующих Супер-ЭВМ с целью повышения производительности;

Научная:

сравнение заявленных производителем данных производительности Супер-ЭВМ с полученными в результате полевых испытаний.

\section{Заключение}
Резюмируя вышесказанное, можно положительно отозваться о точности и корректности работы написанного скрипта. Акцентировать внимание на значимости сути проверки заявленных производителем мощностей, ввиду высокой степени важности корректности работы Суперкомпьютера в таких областях, как предсказания землетрясений, расшифровка ДНК и фармацевтика. А также, благодаря гибкости используемого подхода, открываются перспективы перехода с узконаправленного действия по Суперкомпьютерам в сторону обычных серверов или персональных компьютеров.

\bibliographystyle{alpha}
\bibliography{sample}

\section{Источники}
1. Множественная регрессия в Python [Электронный ресурс].
   - Режим доступа: https://www.delftstack.com/ru/howto/python/perform-multiple-linear-regression-python/

2. Полное руководство по линейной регрессии в Python [Электронный ресурс].
   - Режим доступа: https://www.codecamp.ru/blog/linear-regression-python//

3. Линейная регрессия и основные библиотеки Python для анализа данных и научных вычислений [Электронный ресурс].
   - Режим доступа: https://notebook.community/agushman/coursera/src/cours_2/week_1/peer_review_linreg_height_weight /

4. B.M. Glinskiy, I.M. Kulikov, I.G. Chernykh, A.V. Snytnikov, A.V. Sapetina, D.V. Weins.  The Integrated Approach to Solving Large-Size Physical Problems on Supercomputers.
   Принято к публикации в Communications in Computer and Information Science, изд-во Springer, 2017 г.

5. М.А.Боронина, А.В.Снытников. Разработка высокомасштабируемого параллельного алгоритма для моделирования динамики плазмы.
   //тезисы Международной конференции "Вычислительная и прикладная математика 2017".

6. A.V.Snytnikov.  Porting a Plasma Simulation PIC code from GPU cluster to a cluster built with Intel Xeon Phi accelerators./
   //в печати -  Parallel Computing.
\end{document}
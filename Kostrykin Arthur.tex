\documentclass{article}

% Language setting
% Replace `english' with e.g. `spanish' to change the document language
\usepackage[english,russian]{babel}

% Set page size and margins
% Replace `letterpaper' with `a4paper' for UK/EU standard size
\usepackage[letterpaper,top=2cm,bottom=2cm,left=3cm,right=3cm,marginparwidth=1.75cm]{geometry}

% Useful packages
\usepackage{amsmath}
\usepackage{graphicx}
\usepackage[colorlinks=true, allcolors=blue]{hyperref}

\title{Анализ производительности оборудования супер ЭВМ на основе программы, реализующей метод частиц в ячейках в физике плазмы}
\author{Кострыкин Артур Павлович 22 ВТм}

\begin{document}
\maketitle

\begin{abstract}
Your abstract.
\end{abstract}

\section{Введение}
\section{Актуальность}
\section{Новизна}
\section{Обзор литературы}

\section{Описание работы}
\section{Практическая ценность}
\section{Заключение}


\bibliographystyle{alpha}

\bibliography{sample}

1.B.M. Glinskiy, I.M. Kulikov, I.G. Chernykh, A.V. Snytnikov, A.V. Sapetina, D.V. Weins.  The Integrated Approach to Solving Large-Size Physical Problems on Supercomputers.Принято к публикации в Communications in Computer and Information Science, изд-во Springer, 2017 г.;

2. М.А.Боронина, А.В.Снытников. Разработка высокомасштабируемого параллельного алгоритма для моделирования динамики плазмы;

3. A.V.Snytnikov.  Porting a Plasma Simulation PIC code from GPU cluster to a cluster built with Intel Xeon Phi accelerators./;

4. Статистика о производительности  супер ЭВМ [Электронный ресурс].- Режим доступа: https://www.top500.org/;

5. Введение в использование MPI [Электронный ресурс]. Режим доступа: http://nusc.nsu.ru/wiki/doku.php/doc/mpi/mpi

6. Glinskiy B., Kulikov I., Snytnikov A., et al. Co-design of parallel numerical methods for plasma physics and astrophysics. Supercomputing Frontier [Электронный реусурс]. Режим доступа: http://superfri.org/superfri/;

7. Академия. Эксафлоп/с: почему и как [Электронный ресурс].
Режим доступа: \href{https://www.academia.edu/95135711/Exaflop_s_The_why_and_the_how}{https://www.academia.edu/95135711/Exaflop\_s\_The\_why\_and\_the\_how};

8. Экзафлопсные вычисления и большие данные [Электронный реусурс]. Режим доступа: 
https://www.researchgate.net/publication/282301373\_Exascale\_Computing\_and\_Big\_Data;

9.  Библеотека Python os - Разные интерфейсы операционной системы [Электронный ресурс]. Режим доступа: https://translated.turbopages.org/proxy_u/en-ru.ru.4b04840d-651949ed-f730bcf4-74722d776562/https/docs.python.org/3.8/library/os.html;

10.  Библеотека Python platform - Access to underlying platform’s identifying data [Электронный ресурс]. Режим доступа: https://docs.python.org/3/library/platform.html;

11. Библеотека Python socket - Low-level networking interface [Электронный ресурс]. Режим доступа: https://docs.python.org/3/library/socket.html

12. Библеотека Python subprocess - Subprocess management  [Электронный ресурс]. Режим доступа: https://docs.python.org/3/library/subprocess.html;

13. Компьютерные системы и алгоритмы космической ситуационной осведомленности: история и будущее развитие [Электронный ресурс].
Режим доступа: \href{https://www.academia.edu/47898656/Computer_Systems_and_Algorithms_for_Space_Situational_Awareness_History_and_Future_Development}{https://www.academia.edu/47898656/Computer\_Systems\_and\_Algorithms\_for\_Space\_Situational\_Awareness\_History\_and\_Future\_Development};

14. Информация о запуске python- scripta из с++ [Электронный ресурс]. Режим доступа: https://ru.stackoverflow.com/questions/39243/

15. Особенности работы на Супер ЭВМ кластера НГУ[Электронный ресурс]. Режим доступа: https://ssd.sscc.ru/sites/default/files/content/attach/343/pamyatka_po_ispolzovaniyu_klastera_v4.pdf

16. Адаптация параллельного вычислительного алгоритма к архитектуре суперЭВМ на примере моделирования динамики плазмы методом частиц в ячейках [Электронный реусурс]. Режим доступа: https://lib.nsu.ru/xmlui/handle/nsu/13498

17. Plasma Simulations by Example [Электронный ресурс]. Режим доступа: https://www.particleincell.com/plasma-by-example/

18. СУПЕРКОМПЬЮТЕРЫ И ПАРАЛЛЕЛЬНАЯ ОБРАБОТКА
ДАННЫХ [Электронный ресурс]. Режим доступа: https://teach-in.ru/file/synopsis/pdf/supercomputers-and-parallel-data-processing-M.pdf.

19. Как сделать из Python-скрипта исполняемый файл [Электронный ресурс] Режим доступа: https://habr.com/ru/companies/slurm/articles/746622/

20.  Методология энергосберегающего моделирования и оптимизации для научных приложений [Электронный ресурс].
Режим доступа: \href{https://www.semanticscholar.org/paper/E-AMOM\%3A-an-energy-aware-modeling-and-optimization-Lively-Taylor/fa10af546b3fd83df250f7d8ea5c53016c904af6}{https://www.semanticscholar.org/paper/E-AMOM\%3A-an-energy-aware-modeling-and-optimization-Lively-Taylor/fa10af546b3fd83df250f7d8ea5c53016c904af6};

21. Оптимизация энергоэффективности на уровне алгоритма для процессорного элемента CPU-GPU в вычислениях SIMD/SPMD с интенсивным использованием данных [Электронный ресурс].
Режим доступа: \href{https://www.researchgate.net/publication/220378904_Algorithm_level_power_efficiency_optimization_for_CPU-GPU_processing_element_in_data_intensive_SIMDSPMD_computing}{https://www.researchgate.net/publication/220378904\_Algorithm\_level\_power\_efficiency\_optimization\_for\_CPU-GPU\_processing\_element\_in\_data\_intensive\_SIMDSPMD\_computing};

22. Основы физики плазмы [Электро
нный ресурс]. Режим доступа: https://studfile.net/preview/376604/

23. Концепция масштабирования производительности процессора [Электронный ресурс].
Режим доступа: \href{https://www.kernel.org/doc/html/latest/admin-guide/pm/cpufreq.html}{https://www.kernel.org/doc/html/latest/admin-guide/pm/cpufreq.html};

24. Понятие Супер ЭВМ [Электронный ресурс]. Режим доступа: http://db4.sbras.ru/elbib/data/show_page.phtml?77+658;

25. Исследование энергии-времени, компромисс в программах MPI в кластере с масштабируемой мощностью [Электронный ресурс].
Режим доступа: \href{https://dkl.cs.arizona.edu/publications/papers/ipdps05.pdf}{https://dkl.cs.arizona.edu/publications/papers/ipdps05.pdf};

26. Birdsall, C. K., A. B. Langdon, and R. B. Short. Plasma Physics via Computer Simulation. Taylor \& Francis, 2004.

27. Hockney, R. W., and J. W. Eastwood. Computer Simulation Using Particles. CRC Press, 1988;

28. Трехмерное моделирование плазмы методом частиц в
ячейках на Intel Xeon Phi: оптимизация вычислений и примеры использования[Электронный ресурс]. Режим доступа: https://ceur-ws.org/Vol-1482/495.pdf;

29. ТЕХНОЛОГИЯ РЕШЕНИЯ ЗАДАЧ ФИЗИКИ ПЛАЗМЫ НА СУПЕР-ЭВМ [Электронный ресурс]. Режим доступа: http://book.sarov.ru/wp-content/uploads/Supercomputing-16-2016-38.pdf;

30. Григорьев Ю.Н., Вшивков В.А. Численные методы “частицы-в-ячейках”, Новосибирск: Наука 2000.

31. Бедсел Ч., Лэнгдон Б. Физика плазмы и математическое моделирование. М: Мир, 1989.

32.  GPU Specs Database [Электронный ресурс]. Режим доступа: https://www.techpowerup.com/gpu-specs/

33. Тесты производительности LINPACK [Электронный ресурс].
Режим доступа: \href{https://ru.wikipedia.org/wiki/\%D0\%A2\%D0\%B5\%D1\%81\%D1\%82\%D1\%8B_\%D0\%BF\%D1\%80\%D0\%BE\%D0\%B8\%D0\%B7\%D0\%B2\%D0\%BE\%D0\%B4\%D0\%B8\%D1\%82\%D0\%B5\%D0\%BB\%D1\%8C\%D0\%BD\%D0\%BE\%D1\%81\%D1\%82\%D0\%B8_LINPACK}{https://ru.wikipedia.org/wiki/Тесты\_производительности\_LINPACK};

34. HPL: Оценка производительности Raspberry Pi [Электронный ресурс]. Режим доступа: \href{https://drach.pro/blog/linux/item/88-hpl-raspberry-pi-stress}{https://drach.pro/blog/linux/item/88-hpl-raspberry-pi-stress};

35. High-Performance Linpack (HPL) benchmarking on UL HPC platform [Электронный ресурс]. Режим доступа: \href{https://ulhpc-tutorials.readthedocs.io/en/latest/parallel/mpi/HPL/}{https://ulhpc-tutorials.readthedocs.io/en/latest/parallel/mpi/HPL/};

36. Кластерная архитектура [Электронный ресурс].
   - Режим доступа: https://siblec.ru/telekommunikatsii/vychislitelnye-sistemy-seti-i-telekommunikatsii/5-arkhitektury-vysokoproizvoditelnykh-vychislitelnykh-sistem/5-6-klasternaya-arkhitektura;

37. Параллельные системы [Электронный ресурс].
   - Режим доступа: https://intuit.ru/studies/courses/92/92/lecture/28388?page=3;

38. Состояние и перспективы развития вычислительных систем сверхвысокой производительности [Электронный ресурс].
   - Режим доступа: http://www.botik.ru/PSI/RCMS/publications/publ-texts-2013/Abramov\_6\_22.pdf;

39. Кластер высокопроизводительных вычислений Linux - кластер Беовульф [Электронный ресурс].
   - Режим доступа: https://russianblogs.com/article/7542928214/;

40. Thomas Sterling (computing) [Электронный ресурс].
   - Режим доступа: https://en.m.wikipedia.org/wiki/Thomas\_Sterling\_(computing).































\end{document}
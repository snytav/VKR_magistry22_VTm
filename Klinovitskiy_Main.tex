\documentclass{article}
\usepackage[english,russian]{babel}
\usepackage[letterpaper,top=2cm,bottom=2cm,left=3cm,right=3cm,marginparwidth=1.75cm]{geometry}

% Useful packages
\usepackage{amsmath}
\usepackage{graphicx}
\usepackage[colorlinks=true, allcolors=blue]{hyperref}

\title{РЕГРЕССИОННЫЙ АНАЛИЗ ПРОИЗВОДИТЕЛЬНОСТИ СУПЕР-ЭВМ}
\author{Выполнил студент гр.22-ВТм Клиновицкий Павел}

\begin{document}
\maketitle

\begin{abstract}
Данная работа посвящена вычислению производительности Суперкомпьютера, именуемого далее Супер-ЭВМ, с использованием задач метода частиц в ячейках и последующим расчетам по методам факторного анализа и множественной линейной регрессии при помощи, написанного на языке программирования Python, скрипта, для обработки полученных результатов.

Ключевые слова: Супер-ЭВМ, факторный анализ, множественная линейная регрессия.
\end{abstract}

\section{Введение}
Супер-ЭВМ – это огромная машина, с помощью поддержки векторных операций, способная выполнять 180 миллионов вычислительных операций в секунду над числами с плавающей точкой. В то время, как обычный компьютер имеет один центральный процессор, число ядер которого не превышает и тридцати двух, среднестатистический Суперкомпьютер располагает тысячами, а то и миллионами ядер.

Супер-ЭВМ используется для осуществления трудоемких вычислительных процессов и обработки колоссального объема информации в масштабе реального времени, а также для применения точных моделей исследуемых процессов.

\section{Актуальность}
Необходимость анализа производительности Суперкомпьютера возникает на основании заявленных со стороны производителя мощностей, которые, как правило, не соответствуют реальности на задачах. Полученные в ходе работы и нагрузочных тестирований статистические расчеты предоставят поле для анализа правдоподобности заявлений.
Озвученная проблематика исследовалась как за рубежом, так и со стороны соотечествеников.

\section{Новизна}
Новизна и оригинальность идей, положенных в работу:

Запуск плазменного приложения метода частиц в ячейках, с целью сбора статистической информации о Супер-ЭВМ.

\section{Описание работы}
Написан скрипт для анализов результатов запуска на Суперкомпьютере программы на основе метода частиц в ячейках, исползуемой для моделирования плазмы;

Вычислена производительности Суперкомпьютера при помощи алгоритмов расчетов: факторного анализа, множественной линейной регрессии.

\section{Глава №1 Обзор Литературы}

\section{Глава №2 Компиляция кода Plasma}
Воспользовавшись заготовкой кода из [48], в главе 9: Parallel programming; multithreading, MPI domain decomposition; GPU programming with CUDA, был получен скрипт на языке программирования С++. Данный скрипт был дописан процедурами замера времени выполнения и функцией выгрузки собранных данных в файл, для полчения расчетных данных используемых в дальнейшем в других скриптах.

Планировалось, сравнить полученные данные с данными, полученными в программе Vtune, но сопровождение последней было приостановлено на территории РФ, посему данный момент был опущен.

Про компиляции ряда исполняющих программ в один общий exe файл, возникали ошибки с прочтением директивы mpi.h, создающей конфликтные ситуации на системах Windows.
Проводились попытки установки директивы, установки готовых пакетов инструментов в программе visual studio built. Оказалось, самым лучшим способом выступит переход на систему Linux, где конфликтных ситуаций с директивой mpi.h, не наблюдается.

\section{Глава №3 Анализ}
Для анализа полученных результатов расчетов строится формула, направленная на определение зависимостей производительности ЭВМ  на единицу временного шага в трехмерной сетке: Perf = Nx*Ny*Nz*100*440/t флопс (операция с плавающей точкой в секунду).

Для определения коэффициентов статистического метода оценивания и интерпретации выведенной формулы, используется:
Множественная линейная регрессия – Y = β1x1 + β2x2 +…+ βkxk + U
Где Y-зависимая(объясняемая)переменная, U-случайная составляющая модели, xj-независимые(объясняющие)переменные

Предназначена для проверки и изучения связи (объяснения поведения) между одной зависимой переменной и несколькими независимыми переменными. При использовании статистического метода оценивания и интерпретации, рассчитываются коэффициенты модели по всем показателям.

Отличие между простой и множественной линейной регрессией заключается в том, что вместо линии регрессии в ней используется гиперплоскость.
Преимуществом множественной линейной регрессией, в сравнении с простой линейной регрессией, выступает то, что использование в модели нескольких входных переменных позволяет увеличить долю объяснённой дисперсии выходной переменной, и таким образом улучшить соответствие модели данным. Т.е. при добавлении в модель каждой новой переменной коэффициент детерминации растёт.

Если заваться вопросом, почему выбранная множественная линейная регрессия, а не, к примеру, полиномиальная или опорных векторов, то вот ряд причин:
1. Многогранная аналитика. Благодаря множеству предикторов получается полная картина того, как различные факторы взаимодействуют, влияя на результат. Подобная глубина понимания является ключевой для надежного анализа и эффективного принятия решений;
2. Значимость переменной. Данный метод дает представление о значимости вклада каждого предиктора. С помощью p-значений или доверительных интервалов получается определить, какие переменные оказывают существенное влияние на зависимую переменную;
3. Контролируемый анализ. Способность контролировать мешающие переменные, изолируя уникальное влияние каждого предиктора. Эта возможность гарантирует, что наблюдаемые отношения не являются просто ложными корреляциями.

\section{Научная, научно-техническая и практическая ценность}
Практическая:

повышение качества подбора комплектующих Супер-ЭВМ с целью повышения производительности;

Научная:

сравнение заявленных производителем данных производительности Супер-ЭВМ с полученными в результате полевых испытаний.

\section{Заключение}
Резюмируя вышесказанное, можно положительно отозваться о точности и корректности работы написанного скрипта. Акцентировать внимание на значимости сути проверки заявленных производителем мощностей, ввиду высокой степени важности корректности работы Суперкомпьютера в таких областях, как предсказания землетрясений, расшифровка ДНК и фармацевтика. А также, благодаря гибкости используемого подхода, открываются перспективы перехода с узконаправленного действия по Суперкомпьютерам в сторону обычных серверов или персональных компьютеров.

\bibliographystyle{alpha}
\bibliography{sample}

\section{Источники}
1. Множественная регрессия в Python [Электронный ресурс].
   - Режим доступа: https://www.delftstack.com/ru/howto/python/perform-multiple-linear-regression-python/;

2. Полное руководство по линейной регрессии в Python [Электронный ресурс].
   - Режим доступа: https://www.codecamp.ru/blog/linear-regression-python//;

3. Линейная регрессия и основные библиотеки Python для анализа данных и научных вычислений [Электронный ресурс].
   - Режим доступа: https://notebook.community/agushman/coursera/src/cours_2/week_1/peer_review_linreg_height_weight/;

4. Введение в использование MPI [Электронный ресурс].
   - Режим доступа: http://nusc.nsu.ru/wiki/doku.php/doc/mpi/mpi;

5. https://top50.org [Электронный ресурс];

6. https://top500.org [Электронный ресурс];

7. B.M. Glinskiy, I.M. Kulikov, I.G. Chernykh, A.V. Snytnikov, A.V. Sapetina, D.V. Weins.  The Integrated Approach to Solving Large-Size Physical Problems on Supercomputers.
   Принято к публикации в Communications in Computer and Information Science, изд-во Springer, 2017 г.

8. Академия. Эксафлоп/с: почему и как [Электронный ресурс].
   - Режим доступа: https://www.academia.edu/95135711/Exaflop_s_The_why_and_the_how;

9. Экзафлопсные вычисления и большие данные [Электронный ресурс].
   - https://www.researchgate.net/publication/282301373_Exascale_Computing_and_Big_Data;

10. М.А.Боронина, А.В.Снытников. Разработка высокомасштабируемого параллельного алгоритма для моделирования динамики плазмы.
   //тезисы Международной конференции "Вычислительная и прикладная математика 2017";
   
11. A.V.Snytnikov.  Porting a Plasma Simulation PIC code from GPU cluster to a cluster built with Intel Xeon Phi accelerators./
   //в печати -  Parallel Computing.

12. Параллельные численные методы физики плазмы [Электронный ресурс].
   - Режим доступа: https://superfri.org/index.php/superfri/article/view/26;

13. Компьютерные системы и алгоритмы космической ситуационной осведомленности: история и будущее развитие [Электронный ресурс].
   - Режим доступа: https://www.academia.edu/47898656/Computer_Systems_and_Algorithms_for_Space_Situational_Awareness_History_and_Future_Development;

14. Запуск python-скрипта на c++ [Электронный ресурс].
   - Режим доступа: https://ru.stackoverflow.com/questions/39243/Запуск-python-скрипта-из-c;

15. Памятка по работе с кластером НГУ [Электронный ресурс].
   - Режим доступа: https://ssd.sscc.ru/sites/default/files/content/attach/343/pamyatka_po_ispolzovaniyu_klastera_v4.pdf;

16. Адаптация параллельного вычислительного алгоритма к архитектуре суперЭВМ на примере моделирования динамики плазмы методом частиц в ячейках [Электронный ресурс].
   - Режим доступа: https://lib.nsu.ru/xmlui/handle/nsu/13498;

17. СУПЕРКОМПЬЮТЕРНОЕ СРАВНЕНИЕ ЭФФЕКТИВНОСТИ ИСПОЛЬЗОВАНИЯ РАЗНЫХ МАТЕМАТИЧЕСКИХ ПОСТАНОВОК 3D ГЕОФИЗИЧЕСКОЙ ЗАДАЧИ [Электронный ресурс].
   - Режим доступа: https://bulletin.iis.nsk.su/article/1589;

18. Plasma Simulations by Example [Электронный ресурс].
   - Режим доступа: https://www.particleincell.com/plasma-by-example/;

19. практическое применение DVS для сложных высокопроизводительных вычислений [Электронный ресурс].
   - Режим доступа: https://dkl.cs.arizona.edu/publications/papers/ics09.pdf;

20. Эффективная оценка мощности RTL для крупных проектов [Электронный ресурс].
   - Режим доступа: https://www.computer.org/csdl/proceedings-article/vlsid/2003/18680431/12OmNzvQHNt;

21. методология энергосберегающего моделирования и оптимизации для научных приложений [Электронный ресурс].
   - Режим доступа: https://www.semanticscholar.org/paper/E-AMOM%3A-an-energy-aware-modeling-and-optimization-Lively-Taylor/fa10af546b3fd83df250f7d8ea5c53016c904af6;

22. Оптимизация энергоэффективности на уровне алгоритма для процессорного элемента CPU-GPU в вычислениях SIMD/SPMD с интенсивным использованием данных [Электронный ресурс].
   - Режим доступа: https://www.researchgate.net/publication/220378904_Algorithm_level_power_efficiency_optimization_for_CPU-GPU_processing_element_in_data_intensive_SIMDSPMD_computing;

23. Графический процессор общего назначения (GPGPU) для векторных инструкций [Электронный ресурс].
   - Режим доступа: https://www.sci-hub.ru/10.1016/j.jpdc.2010.10.007;

24. СУПЕРКОМПЬЮТЕРЫ И ПАРАЛЛЕЛЬНАЯ ОБРАБОТКА ДАННЫХ [Электронный ресурс].
   - Режим доступа: https://teach-in.ru/file/synopsis/pdf/supercomputers-and-parallel-data-processing-M.pdf;

25. Концепция масштабирования производительности процессора [Электронный ресурс].
   - Режим доступа: https://www.kernel.org/doc/html/latest/admin-guide/pm/cpufreq.html;

26. Адаптация производительности многопоточных программ, использование аппаратного прогнозирования на основе событий [Электронный ресурс].
   - Режим доступа: https://www.researchgate.net/publication/221235880_Online_power-performance_adaptation_of_multithreaded_programs_using_hardware_event-based_prediction;

27. Сокращение EDP на основе вспомогательного потока, схема адаптации выполнения приложения в CMP [Электронный ресурс].
   - Режим доступа: https://www.researchgate.net/publication/221235880_Online_power-performance_adaptation_of_multithreaded_programs_using_hardware_event-based_prediction;

28. Исследование энергии-времени, компромисс в программах MPI в кластере с масштабируемой мощностью [Электронный ресурс].
   - Режим доступа: https://dkl.cs.arizona.edu/publications/papers/ipdps05.pdf;

29. Энергоэффективная система выполнения для высокопроизводительных вычислений [Электронный ресурс].
   - Режим доступа: https://www.researchgate.net/publication/4204625_A_Power-Aware_Run-Time_System_for_High-Performance_Computing;

31. Физика плазмы с помощью компьютерного моделирования [Электронный ресурс].
   - Режим доступа: https://www.taylorfrancis.com/books/mono/10.1201/9781315275048/plasma-physics-via-computer-simulation-langdon-birdsall;

32. Компьютерное моделирование с использованием частиц [Электронный ресурс].
   - Режим доступа: https://archive.org/details/computersimulati0000hock/page/n5/mode/2up;

33. Численные методы частиц в ячейках [Электронный ресурс].
   - Режим доступа: https://www.degruyter.com/document/doi/10.1515/9783110916706/html;

34. Технология решения задач физики плазмы на суперЭВМ [Электронный ресурс].
   - Режим доступа: http://book.sarov.ru/wp-content/uploads/Supercomputing-16-2016-38.pdf;

35. теоретико-информационный подход к оценке производительности суперкомпьютеров [Электронный ресурс].
   - Режим доступа: https://cyberleninka.ru/article/n/teoretiko-informatsionnyy-podhod-k-otsenke-proizvoditelnosti-superkompyuterov/viewer;

36. Исследование производительности супер-ЭВМ семейства "Скиф Аврора" на индустриальных задачах [Электронный ресурс].
   - Режим доступа: https://mmp.susu.ru/pdf/211/7.pdf;

37. Почему высокопроизводительные вычисления нуждаются в переменах [Электронный ресурс].
   - Режим доступа: https://www.itweek.ru/infrastructure/article/detail.php?ID=226684;

38. Суперкомпьютеры для всех [Электронный ресурс].
   - Режим доступа: https://www.osp.ru/os/2007/08/4493229;

39. Кластерная архитектура [Электронный ресурс].
   - Режим доступа: https://siblec.ru/telekommunikatsii/vychislitelnye-sistemy-seti-i-telekommunikatsii/5-arkhitektury-vysokoproizvoditelnykh-vychislitelnykh-sistem/5-6-klasternaya-arkhitektura;

40. Системы высокопроизводительных вычислений в 2020–2021 годах: обзор достижений и анализ рынков [Электронный ресурс].
   - Режим доступа: http://www.cadcamcae.lv/N145/63-79.pdf;

41. 10 самых мощных суперкомпьютеров мира [Электронный ресурс].
   - Режим доступа: https://naked-science.ru/article/top/10-fastest-supercomputers;

42. Параллельные системы [Электронный ресурс].
   - Режим доступа: https://intuit.ru/studies/courses/92/92/lecture/28388?page=3;

43. Состояние и перспективы развития вычислительных систем сверхвысокой производительности [Электронный ресурс].
   - Режим доступа: http://www.botik.ru/PSI/RCMS/publications/publ-texts-2013/Abramov_6_22.pdf;

44. Лучшие суперкомпьютеры мира — как выглядят и зачем нужны? [Электронный ресурс].
   - Режим доступа: https://habr.com/ru/companies/first/articles/714622/;

45. К точке критического перехода [Электронный ресурс].
   - Режим доступа: https://3dnews.ru/645889;

46. Кластер высокопроизводительных вычислений Linux - кластер Беовульф [Электронный ресурс].
   - Режим доступа: https://russianblogs.com/article/7542928214/;

47. Thomas Sterling (computing) [Электронный ресурс].
   - Режим доступа: https://en.m.wikipedia.org/wiki/Thomas_Sterling_(computing);
   
48. Plasma Simulations by Example [Электронный ресурс].
   - Режим доступа: https://www.particleincell.com/plasma-by-example/;

\end{document}
\documentclass{article}
\usepackage[english,russian]{babel}
\usepackage[letterpaper,top=2cm,bottom=2cm,left=3cm,right=3cm,marginparwidth=1.75cm]{geometry}

% Useful packages
\usepackage{amsmath}
\usepackage{graphicx}
\usepackage[colorlinks=true, allcolors=blue]{hyperref}

\title{РЕГРЕССИОННЫЙ АНАЛИЗ ПРОИЗВОДИТЕЛЬНОСТИ СУПЕР-ЭВМ}
\author{Выполнил ученик гр.22-ВТм Клиновицкий павел}

\begin{document}
\maketitle

\begin{abstract}
Данная статья посвящена вопросам вычисления производительности Суперкомпьютера, он же Супер-ЭВМ, с использованием задач метода частиц в ячейках и последующим расчетам по методам факторного анализа и множественной линейной регрессии при помощи, написанного на языке программирования Python, скрипта, для обработки полученных результатов.

Ключевые слова: Супер-ЭВМ, факторный анализ, множественная линейная регрессия.
\end{abstract}

\section{Введение}
Суперкомпьютер, он же Супер-ЭВМ – это огромная машина, выполняющая триллионы вычислительных операций. В то время, как обычный компьютер имеет один центральный процессор, число ядер которого не превышает и тридцати двух, среднестатистический Суперкомпьютер располагает тысячами, а то и миллионами ядер.

Супер-ЭВМ используется для осуществления трудоемких вычислительных процессов и обработки колоссального объема информации в масштабе реального времени, а также для применения точных моделей исследуемых процессов.

Необходимость анализа производительности Суперкомпьютера возникает на основании заявленных со стороны производителя мощностей, обозначенных в рейтингах top50 и top500, которые, как правило, не соответствуют действительности. Полученные в ходе работы и нагрузочных тестирований статистические расчеты предоставят поле для анализа правдоподобности заявлений.

\section{Актуальность}
\section{Новизна}

\section{Обзор литературы}
В работе \cite{greenwade93}

\section{Описание работы}
//Описание работы//

\section{Научная, научно-техническая и практическая ценность}
Практическая:

повышение качества подбора комплектующих Супер-ЭВМ с целью повышения производительности;

Научная:

сравнение заявленных производителем данных производительности Супер-ЭВМ с полученными в результате полевых испытаний.

\section{Заключение}
Резюмируя вышесказанное, можно положительно отозваться о точности и корректности работы написанного скрипта. Акцентировать внимание на значимости сути проверки заявленных производителем мощностей, ввиду высокой степени важности корректности работы Суперкомпьютера в таких областях, как предсказания землетрясений, расшифровка ДНК и фармацевтика. А также, благодаря гибкости используемого подхода, открываются перспективы перехода с узконаправленного действия по Суперкомпьютерам в сторону обычных серверов или персональных компьютеров.

\bibliographystyle{alpha}
\bibliography{sample}

\end{document}